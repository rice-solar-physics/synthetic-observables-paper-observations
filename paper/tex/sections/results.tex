%%%%%%%%%%%%%%%%%%%%%%%%%%%%%%%%%%%%%%%%%%%%%%%%%%%%%%%%%%%%%%%%%%%%%%%%%%%%%%%
%                                   Timelags                                  %
%%%%%%%%%%%%%%%%%%%%%%%%%%%%%%%%%%%%%%%%%%%%%%%%%%%%%%%%%%%%%%%%%%%%%%%%%%%%%%%
\section{Timelags}\label{timelags}

\begin{figure*}
    %\plotone{../figures/observed_timelags.pdf}
    \caption{Timelag maps as calculated from intensity data from observations of \AR{} NOAA 1158 by SDO/AIA. Timelag maps are shown for every possible channel pair as indicated in the upper left corner of each map. As in \autoref{fig:model_timelags}, the colorbar range from -5000 s to +5000 s.}
    \label{fig:observed_timelags}
\end{figure*}

\begin{figure*}
    %\plotone{../figures/observed_correlations.pdf}
    \caption{Same as \autoref{fig:observed_timelags} except here we show the maximum value of the cross-correlation as derived from the observations.}
    \label{fig:observed_correlations}
\end{figure*}

\authorcomment2{Discussion of observed timelags and cross-correlation values}

\authorcomment2{Possibly put histograms of observations versus models here too}

\authorcomment1{Emphasize the point that no single heating model is consistent with the observations. Need multiple heating frequencies; this will lead into random forest stuff}

%%%%%%%%%%%%%%%%%%%%%%%%%%%%%%%%%%%%%%%%%%%%%%%%%%%%%%%%%%%%%%%%%%%%%%%%%%%%%%%
%                                   EM Slopes                                 %
%%%%%%%%%%%%%%%%%%%%%%%%%%%%%%%%%%%%%%%%%%%%%%%%%%%%%%%%%%%%%%%%%%%%%%%%%%%%%%%
\section{Emission Measure Slopes}\label{em_slopes}

In addition to the timelag, we also compute the emission measure distribution (\dem) using the method of \citet{hannah_differential_2012} using both our simulated and observed AIA data. \dem provides a measure of how much plasma is emitting over a range of temperatures and has been used by many workers to assess heating frequency in \AR{} cores \citep[][and references therein]{tripathi_emission_2011,warren_constraints_2011,warren_systematic_2012,schmelz_cold_2012,bradshaw_diagnosing_2012,reep_diagnosing_2013,barnes_inference_2016,barnes_inference_2016,barnes_inference_2016-1}. 

A common technique is to fit a power-law $\mathrm{EM}(T)\sim T^a$ to \dem between approximately 1 and 4 MK and is effectively a measure of how isothermal the distribution is. Many workers \citep[see Table 3 of][and references therein]{bradshaw_diagnosing_2012} have found $2<a<5$. \citet{cargill_active_2014} used a range of waiting times dependent on the event energy (see \autoref{heating}) and found that he could account for the observed range of slopes. Thus, the \dem slope $a$ is one important diagnostic for assessing how often a single loop may be reheated. We compute $a$ between $10^6$ and $10^{6.6}$ K for each pixel in our \AR{}.

%%%%%%%%%%%%%%%%%%%%%%%%%%%%%%%%%%%%%%%%%%%%%%%%%%%%%%%%%%%%%%%%%%%%%%%%%%%%%%%
%                                   Pixel Classification                      %
%%%%%%%%%%%%%%%%%%%%%%%%%%%%%%%%%%%%%%%%%%%%%%%%%%%%%%%%%%%%%%%%%%%%%%%%%%%%%%%
\section{Pixel Classification}\label{classify}

\authorcomment1{Describe random forest technique; how is data prepared; what is the RF actually doing}

\begin{figure*}
    %\plotone{../figures/probability_maps.pdf}
    \caption{Foo bar}
    \label{fig:probability_maps}
\end{figure*}

\authorcomment1{Describe and interpret the results of the classification}

\begin{figure}
    %\plotone{../figures/frequency_map.pdf}
    \caption{foo bar}
    \label{fig:frequency_map}
\end{figure}