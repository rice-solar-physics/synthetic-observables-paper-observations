%% Use AASTeX class, version 6.1
%% Allow for additional class options such as,
%%  twocolumn   : two text columns, 10 point font, single spaced article.
%%                This is the most compact and represent the final published
%%                derived PDF copy of the accepted manuscript from the publisher
%%  manuscript  : one text column, 12 point font, double spaced article.
%%  preprint    : one text column, 12 point font, single spaced article.  
%%  preprint2   : two text columns, 12 point font, single spaced article.
%%  modern      : a stylish, single text column, 12 point font, article with
%% 		            wider left and right margins. This uses the Daniel
%% 		            Foreman-Mackey and David Hogg design.
%%  astrosymb    : Loads Astrosymb font and define \astrocommands. 
%%  tighten      : Makes baselineskip slightly smaller, only works with 
%%                 the twocolumn substyle.
%%  times        : uses times font instead of the default
%%  linenumbers  : turn on lineno package.
%%  trackchanges : required to see the revision mark up and print its output
%%  longauthor   : Do not use the more compressed footnote style (default) for 
%%                 the author/collaboration/affiliations. Instead print all
%%                 affiliation information after each name. Creates a much
%%                 long author list but may be desirable for short author papers

\documentclass[preprint2,linenumbers]{aastex62}
%% Include packages
\usepackage{amsmath}
\usepackage{calc}
\usepackage{pythontex}
% % % % % % % % % % % % % % % % % % % % % % % % % % % % % % % % % %
% PythonTeX Bug Fix % % % % % % % % % % % % % % % % % % % % % % % %
% % % % % % % % % % % % % % % % % % % % % % % % % % % % % % % % % % 
% pytexbug fix for context in customcode.
\makeatletter
\renewenvironment{pythontexcustomcode}[2][begin]{%
	\VerbatimEnvironment
	\Depythontex{env:pythontexcustomcode:om:n}%
	\ifstrequal{#1}{begin}{}{%
		\ifstrequal{#1}{end}{}{\PackageError{\pytx@packagename}%
			{Invalid optional argument for pythontexcustomcode}{}
		}%
	}%
	\xdef\pytx@type{CC:#2:#1}%
	\edef\pytx@cmd{code}%
	% PATCH \def\pytx@context{}%
	\pytx@SetContext
	% END PATCH
	\def\pytx@group{none}%
	\pytx@BeginCodeEnv[none]}%
{\end{VerbatimOut}%
\setcounter{FancyVerbLine}{\value{pytx@FancyVerbLineTemp}}%
\stepcounter{\pytx@counter}%
}%
\makeatother
% % % % % % % % % % % % % % % % % % % % % % % % % % % % % % % % % %
\setpythontexcontext{textwidth=\the\textwidth,columnwidth=\the\columnwidth,figurewidth=\the\columnwidth}
%% Custom commands
\DeclareMathOperator*{\argmax}{arg\,max} % in your preamble
\newcommand{\AR}{active region}
\newcommand{\dem}{$\mathrm{EM}(T)$}
\renewcommand{\sectionautorefname}{Section}
\renewcommand{\subsectionautorefname}{Section}
\renewcommand{\subsubsectionautorefname}{Section}
%% Paper Aliases
\defcitealias{barnes_understanding_2018}{Paper I}
%%%%%%%%%%%%%%%%%%%%%%%%%%%%%%%%%%%%%%%%%%%%%%%%%%%%%%%%%%%%%%%%%%%%%%%%%%%%%%%
%                                   Body                                      %
%%%%%%%%%%%%%%%%%%%%%%%%%%%%%%%%%%%%%%%%%%%%%%%%%%%%%%%%%%%%%%%%%%%%%%%%%%%%%%%
\begin{document}
% TeXFigure Manager
\begin{pythontexcustomcode}{py}
# Imports
import os
import texfigure
pytex.formatter = texfigure.repr_latex_formatter
import numpy as np
import matplotlib.pyplot as plt
import matplotlib.colors
from sunpy.map import Map
import astropy.units as u
from astropy.coordinates import SkyCoord
from astropy.visualization import ImageNormalize,AsinhStretch,SqrtStretch
import synthesizAR
# Set some plot styling here
plt.rcParams['font.family'] = 'serif'
plt.rcParams['font.serif'] = 'Computer Modern'
plt.rcParams['text.usetex'] = True
plt.rcParams['axes.titlesize'] = 10
plt.rcParams['axes.labelsize'] = 10
plt.rcParams['legend.fontsize'] = 8
plt.rcParams['xtick.labelsize'] = 8
plt.rcParams['ytick.labelsize'] = 8
plt.rcParams['xtick.major.pad'] = 8
plt.rcParams['xtick.minor.pad'] = 8
plt.rcParams['ytick.major.pad'] = 8
plt.rcParams['ytick.minor.pad'] = 8
plt.rcParams['xtick.direction'] = 'in'
plt.rcParams['ytick.direction'] = 'in'
plt.rcParams['savefig.dpi'] = 200
plt.rcParams['savefig.format'] = 'pdf'
plt.rcParams['savefig.bbox'] = 'tight'
# Some useful quantities
channels = [94,131,171,193,211,335]
heating = ['high_frequency', 'intermediate_frequency','low_frequency']
channel_pairs = [(94,335), (94,171), (94,193),(94,131),(94,211),(335,131),(335,193),
                 (335,211),(335,171),(211,131),(211,171),(211,193),(193,171),(193,131),
                 (171,131),]
# Reset LaTeX repr for figures
fig_str = r"""
\begin{{{figure_env_name}}}
    \centering
    {myfig}
    \caption{{{caption}}}
    \label{{{label}}}
\end{{{figure_env_name}}}
"""
\end{pythontexcustomcode}
%%%%%%%%%%%%%%%%%%%%%%%%%%%%%%%%%%%%%%%%%%%%%%%%%%%%%%%%%%%%%%%%%%%%%%%%%%%%%%%
%                                   Title and Authors                         %
%%%%%%%%%%%%%%%%%%%%%%%%%%%%%%%%%%%%%%%%%%%%%%%%%%%%%%%%%%%%%%%%%%%%%%%%%%%%%%%
\title{Understanding Heating Frequency in Active Region Cores through Synthetic Observables II. Classifying Observations}
\author[0000-0001-9642-6089]{W. T. Barnes}
\author{S. J. Bradshaw}
\affiliation{Department of Physics \& Astronomy, Rice University, Houston, TX 77005-1827}
\author{N. M. Viall}
\affiliation{NASA Goddard Space Flight Center, Greenbelt, MD 20771}
\correspondingauthor{W. T. Barnes}
\email{will.t.barnes@rice.edu}
%%%%%%%%%%%%%%%%%%%%%%%%%%%%%%%%%%%%%%%%%%%%%%%%%%%%%%%%%%%%%%%%%%%%%%%%%%%%%%%
%                                   Abstract                                  %
%%%%%%%%%%%%%%%%%%%%%%%%%%%%%%%%%%%%%%%%%%%%%%%%%%%%%%%%%%%%%%%%%%%%%%%%%%%%%%%
\begin{abstract}
The abstract will go here.
\end{abstract}
%% Keywords
\keywords{Sun,corona,nanoflares,active regions}
%%%%%%%%%%%%%%%%%%%%%%%%%%%%%%%%%%%%%%%%%%%%%%%%%%%%%%%%%%%%%%%%%%%%%%%%%%%%%%%
%                                   Sections                                  %
%%%%%%%%%%%%%%%%%%%%%%%%%%%%%%%%%%%%%%%%%%%%%%%%%%%%%%%%%%%%%%%%%%%%%%%%%%%%%%%
%%%%%%%%%%%%%%%%%%%%%%%%%%%%%%%%%%%%%%%%%%%%%%%%%%%%%%%%%%%%%%%%%%%%%%%%%%%%%%%
%                                   Introduction                              %
%%%%%%%%%%%%%%%%%%%%%%%%%%%%%%%%%%%%%%%%%%%%%%%%%%%%%%%%%%%%%%%%%%%%%%%%%%%%%%%
\section{Introduction}\label{introduction}

\authorcomment1{Brief summary of Paper 1; emphasize need for detailed comparisons with observations; summarize what we will do in this paper; this can be relatively short because the main intro is provided in paper 1}

%%%%%%%%%%%%%%%%%%%%%%%%%%%%%%%%%%%%%%%%%%%%%%%%%%%%%%%%%%%%%%%%%%%%%%%%%%%%%%%
%                                   Data                                      %
%%%%%%%%%%%%%%%%%%%%%%%%%%%%%%%%%%%%%%%%%%%%%%%%%%%%%%%%%%%%%%%%%%%%%%%%%%%%%%%
\section{Observations and Analysis}\label{observations}

\begin{pycode}[manager_data]
manager_data = texfigure.Manager(
    pytex, './',
    python_dir='python',
    fig_dir='figures',
    data_dir='data',
    number=1,
)
\end{pycode}

We analyze 12 hours of AIA observations of \AR{} NOAA 1158 in six EUV channels, 94, 131, 171, 193, 211, and 335 \AA{}, beginning at 2011 February 12 12:00:00 UTC and ending at 2011 February 13 00:00:00 UTC. The \AR{} was chosen from the catalogue of \AR{} s originally compiled by \citet{warren_systematic_2012} and was also studied by \citet{viall_survey_2017}. The full-disk, level-1 AIA data products in FITS file format are obtained from the Joint Science Operations Center \citep[JSOC,][]{couvidat_observables_2016} archive at the full instrument cadence of 12 s and the full spatial resolution of 0.6\arcsec-per-pixel using the drms Python client \authorcomment1{Need Zenodo DOI}. This amounts to a total of 21597 images across all six channels and the entire observing window.

After downloading the data, we apply the \texttt{aiaprep} method, as implemented in SunPy \citep{sunpy_community_sunpypython_2015}, to each full-disk image in order to process the level-1 data into a level-1.5 data. Next, we align each image with the observation at 2011 February 12 15:33:45 UTC (the time of the original observation of NOAA 1158 by \citet{warren_systematic_2012}) by ``derotating'' each image using the Snodgrass empirical rotation rate \citep{snodgrass_magnetic_1983}. After aligning the images in every channel to a common time, we crop each full-disk image such that the bottom left corner of the image is $(-440\arcsec,-375\arcsec)$ and the top right corner is $(-140\arcsec,-75\arcsec)$, where the two coordinates are the longitude and latitude, respectively, in the helioprojective coordinate system \citep[see][]{thompson_coordinate_2006} defined by an observer at the location of the SDO spacecraft on 2011 February 12 15:33:45. \autoref{fig:intensity-maps} shows the level-1.5, derotated and cropped AIA observations of \AR{} NOAA 1158 at 2011 February 12 15:33:45 in all six EUV channels of interest.

\begin{pycode}[manager_data]
fits_files = [
    'aia_lev1.5_20110212T153238_94_cutout.fits',
    'aia_lev1.5_20110212T153245_131_cutout.fits',
    'aia_lev1.5_20110212T153248_171_cutout.fits',
    'aia_lev1.5_20110212T153243_193_cutout.fits',
    'aia_lev1.5_20110212T153248_211_cutout.fits',
    'aia_lev1.5_20110212T153239_335_cutout.fits',
]
fig = plt.figure(figsize=texfigure.figsize(
    pytex,
    scale=1 if is_onecolumn() else 2,
    height_ratio=2.25/3,
    figure_width_context='columnwidth'
))
plt.subplots_adjust(hspace=0.05,wspace=0.03)
for i,f in enumerate(fits_files):
    m = Map(os.path.join(manager_data.data_dir, 'observations', f))
    m = Map(m.data/m.meta['exptime'], m.meta)
    m = m.submap(SkyCoord(Tx=-440*u.arcsec,Ty=-380*u.arcsec,frame=m.coordinate_frame),
                 SkyCoord(Tx=-185*u.arcsec,Ty=-125*u.arcsec,frame=m.coordinate_frame))
    ax = fig.add_subplot(2,3,i+1,projection=m)
    norm = ImageNormalize(vmin=0, vmax=m.data.max(), stretch=SqrtStretch())
    im = m.plot(axes=ax, title=False, annotate=False, norm=norm)
    ax.grid(alpha=0)
    lon,lat = ax.coords
    lon.set_ticks(color='k',number=4)
    lat.set_ticks(color='k',number=4)
    if i != 3:
        lon.set_ticklabel_visible(False)
        lat.set_ticklabel_visible(False)
    else:
        lat.set_ticklabel(rotation='vertical')
        lon.set_axislabel('Helioprojective Longitude')
        lat.set_axislabel('Helioprojective Latitude')
    xtext,ytext = m.world_to_pixel(SkyCoord(-420*u.arcsec, -150*u.arcsec, frame=m.coordinate_frame))
    ax.text(
        xtext.value, ytext.value,
        f'{m.meta["wavelnth"]} $\mathrm{{\AA}}$',
        color='w',
        fontsize=plt.rcParams['legend.fontsize']
    )
    pos = ax.get_position().get_points()
    cax = fig.add_axes([pos[0,0], pos[1,1]+0.0075, pos[1,0]-pos[0,0], 0.015])
    cbar = fig.colorbar(im, cax=cax, orientation='horizontal')
    cbar.locator = MaxNLocator(nbins=4, prune='lower')
    cbar.ax.tick_params(labelsize=plt.rcParams['legend.fontsize'], width=0.5)
    cbar.update_ticks()
    cbar.ax.xaxis.set_ticks_position('top')
    cbar.outline.set_linewidth(0.5)

fig_intensity_maps = manager_data.save_figure('intensity-maps')
fig_intensity_maps.caption = r'Active region NOAA 1158 as observed by AIA on 2011 February 12 15:32 UTC in the six EUV channels of interest. The data have been processed to level-1.5, aligned to the image at 2011 February 12 15:33:45 UTC, and cropped to the area surrounding NOAA 1158. The intensities are in units of DN pixel$^{-1}$ s$^{-1}$. In each image, the colorbar is on a square root scale and is normalized between zero and the maximum intensity. The color tables are the standard AIA color tables as implemented in SunPy.'
fig_intensity_maps.figure_env_name = 'figure*'
fig_intensity_maps.figure_width = r'\columnwidth' if is_onecolumn() else r'2\columnwidth'
fig_intensity_maps.fig_str = fig_str
\end{pycode}
\py[manager_data]|fig_intensity_maps|

%%%%%%%%%%%%%%%%%%%%%%%%%%%%%%%%%%%%%%%%%%%%%%%%%%%%%%%%%%%%%%%%%%%%%%%%%%%%%%%
%                                   EM Slopes                                 %
%%%%%%%%%%%%%%%%%%%%%%%%%%%%%%%%%%%%%%%%%%%%%%%%%%%%%%%%%%%%%%%%%%%%%%%%%%%%%%%
\subsection{Emission Measure Slopes}\label{em_slopes}

\begin{pycode}[manager_em]
manager_em = texfigure.Manager(
    pytex, './',
    python_dir='python',
    fig_dir='figures',
    data_dir='data',
    number=2,
)
from formatting import heating_palette
\end{pycode}

After prepping, aligning, and cropping all 12 hours of AIA data for all six channels, we carry out the same analysis that we applied to our predicted observations in \citetalias{barnes_understanding_2019} in order to compute two diagnostics of the heating: the emission measure slope and the timelag. First, we compute the emission measure distribution, \dem, in each pixel of the \AR{} from the time-averaged intensities from all six channels using the regularized inversion method of \citet{hannah_differential_2012}. As in \citetalias{barnes_understanding_2019}, we use temperature bins of width $\Delta\log T=0.1$ with the left and right edges at $10^{5.5}$ K and $10^{7.2}$ K, respectively. The uncertainties on the intensities are estimated using \texttt{aia\_bp\_estimate\_error.pro} procedure provided by the AIA instrument team in the SolarSoftware package \citep[SSW,][]{freeland_data_1998}. 

As we noted in the introduction to \citetalias{barnes_understanding_2019}, the emission measure distribution is well-described by the power-law relationship $\textup{EM}(T)\sim T^a$, for $a>0$, over the temperature range $10^{5.5}\lesssim T\lesssim10^{6.5}$ K \citep{jordan_structure_1975,jordan_structure_1976} where $a$ is the emission measure slope. $a$ parameterizes the width of the emission measure distribution and as such is a commonly used diagnostic for the heating frequency \citep[e.g.][]{tripathi_emission_2011,winebarger_using_2011,warren_constraints_2011,mulu-moore_can_2011,bradshaw_diagnosing_2012,schmelz_cold_2012,reep_diagnosing_2013,cargill_active_2014,del_zanna_evolution_2015}. 

\begin{pycode}[manager_em]
fig = plt.figure(figsize=texfigure.figsize(
    pytex,
    scale=1 if is_onecolumn() else 2,
    height_ratio=1/2,
    figure_width_context='columnwidth'
))
plt.subplots_adjust(wspace=0.31)
### Map ###
slope_map = Map(os.path.join(manager_em.data_dir, 'observations', 'em_slope.fits'))
rsquared_map = Map(os.path.join(manager_em.data_dir, 'observations', 'em_slope_rsquared.fits'))
slope_map = Map(slope_map.data, slope_map.meta, mask=rsquared_map.data < rsquared_threshold)
slope_map = slope_map.submap(
    SkyCoord(Tx=-410*u.arcsec,Ty=-325*u.arcsec,frame=slope_map.coordinate_frame),
    SkyCoord(Tx=-225*u.arcsec,Ty=-150*u.arcsec,frame=slope_map.coordinate_frame))
ax = fig.add_subplot(121, projection=slope_map)
im = slope_map.plot(
    axes=ax,
    cmap='viridis',
    vmin=1.5,
    vmax=5.5,
    title=False,
    annotate=False
)
ax.grid(alpha=0)
# HPC Axes
lon,lat = ax.coords
lon.set_ticks(number=4)
lat.set_ticks(number=2)
lat.set_ticklabel(rotation='vertical')
lon.set_axislabel('Helioprojective Longitude',)
lat.set_axislabel('Helioprojective Latitude',)
# Colorbar
pos = ax.get_position().get_points()
cax = fig.add_axes([pos[0,0], pos[1,1]+0.01, pos[1,0]-pos[0,0], 0.025])
cbar = fig.colorbar(im,cax=cax, orientation='horizontal',)
cbar.ax.xaxis.set_ticks_position('top')
cbar.set_ticks([2,3,4,5])
### Histograms ###
ax = fig.add_subplot(122,)
bins = np.arange(0, 8, 0.05)
colors = ['k'] + heating_palette()
# Plot Model EM Slopes
for i,h in enumerate(['observations'] + heating):
    m = Map(os.path.join(manager_em.data_dir, f'{h}', 'em_slope.fits'))
    m_rsquared = Map(os.path.join(manager_em.data_dir, f'{h}', 'em_slope_rsquared.fits'))
    m = Map(m.data, m.meta, mask=m_rsquared.data < rsquared_threshold)
    h,b,_ = ax.hist(
        m.data[~m.mask],
        bins='fd',#bins,
        histtype='step',
        density=True,
        color=colors[i],
        label=h.split('_')[0].capitalize()
    )
# Ticks and Spines
ax.set_xlim(1, 8);
ax.xaxis.set_major_locator(FixedLocator([2, 3, 4, 5, 6, 7]))
ax.yaxis.set_major_locator(MaxNLocator(nbins=7, prune='both'))
ax.spines['top'].set_visible(False)
ax.spines['right'].set_visible(False)
ax.spines['left'].set_bounds(ax.get_yticks()[0],ax.get_yticks()[-1])
ax.spines['bottom'].set_bounds(ax.get_xticks()[0],ax.get_xticks()[-1])
# Labels and legends
ax.set_xlabel(r'$a$')
ax.set_ylabel(r'Number of Pixels (Normalized)')
ax.legend(frameon=False,loc=1)
### Save ###
fig_em_slopes = manager_em.save_figure('em-slopes')
fig_em_slopes.caption = r'\textit{Left:} Map of emission measure slope, $a$, in each pixel of \AR{} NOAA 1158. The \dem{} is computed from the observed AIA intensities in the six EUV channels time-averaged over the 12-hour observing window using the method of \citet{hannah_differential_2012}. The \dem{} in each pixel is then fit to $T^a$ over the temperature interval $8\times10^5\,\textup{K}\le T < T_{peak}$. Any pixels with $r^2<0.75$ are masked and colored white. \textit{Right:} Distribution of emission measure slopes computed from the observed intensities (black) and the predicted intensities from \citetalias{barnes_understanding_2019} (blue, orange, green). In each case, the bins are determined using the Freedman Diaconis estimator \citep{freedman_histogram_1981} as implemented in the Numpy package for array computation in Python \citep{oliphant_guide_2006} and each histogram is normalized such that the area under the histogram is equal to 1.'
fig_em_slopes.figure_env_name = 'figure*'
fig_em_slopes.figure_width = r'\columnwidth' if is_onecolumn() else r'2\columnwidth'
fig_em_slopes.fig_str = fig_str
\end{pycode}
\py[manager_em]|fig_em_slopes|

The left panel of \autoref{fig:em-slopes} shows the emission measure slope, $a$, as computed from the emission measure distribution in each pixel of \AR{} NOAA 1158. We calculate $a$ by fitting a first-order polynomial to the log-transformed emission measure and the temperature bin centers, $\log_{10}\textup{EM}\sim a\log_{10}T$. As in \citetalias{barnes_understanding_2019}, the fit is only computed over the temperature range $8\times10^5\,\textup{K}\le T < T_{peak}$, where $T_{peak}=\argmax_T\,\textup{EM}(T)$ is the temperature at which the emission measure distribution peaks. If $r^2<\py|rsquared_threshold|$ in any pixel, where $r^2$ is the correlation coefficient for the first-order polynomial fit, the pixel is masked and colored white.

The right panel of \autoref{fig:em-slopes} shows the distribution of emission measure slopes for every pixel in the \AR{} where $r^2\ge\py|rsquared_threshold|$. As noted in the legend, the black histogram denotes the slopes computed from the real AIA observations while the blue, orange, and green histograms are the distributions of emission measure slopes computed from the predicted AIA intensities in \citetalias{barnes_understanding_2019} for high-, intermediate-, and low-frequency nanoflares, respectively. Note that the observed distribution of slopes overlaps the distributions of predicted slopes for all three heating scenarios, suggesting that a range of nanoflare heating frequencies is operating across the \AR.

%%%%%%%%%%%%%%%%%%%%%%%%%%%%%%%%%%%%%%%%%%%%%%%%%%%%%%%%%%%%%%%%%%%%%%%%%%%%%%%
%                                   Timelags                                  %
%%%%%%%%%%%%%%%%%%%%%%%%%%%%%%%%%%%%%%%%%%%%%%%%%%%%%%%%%%%%%%%%%%%%%%%%%%%%%%%
\subsection{Timelags}\label{timelags}

\begin{pycode}[manager_timelags]
manager_timelags = texfigure.Manager(
    pytex, './',
    python_dir='python',
    fig_dir='figures',
    data_dir='data',
    number=3,
)
file_format = os.path.join(manager_timelags.data_dir, 'observations', '{}_{}_{}.fits')
from synthesizAR.visualize import bgry_004_idl_cmap
\end{pycode}

% Discussion of observed timelags and cross-correlation values
% Possibly put histograms of observations versus models here too
% Emphasize the point that no single heating model is consistent with the observations. Need multiple heating frequencies; this will lead into random forest stuff

Next, we apply the timelag analysis of \citet{viall_evidence_2012} to every pixel in the \AR{} over the entire 12-hour observing window at the full temporal and spatial resolution. As in \citetalias{barnes_understanding_2019}, we compute the cross-correlation, $\mathcal{C}_{AB}$, between all possible pairs, $AB$, of the six EUV channels of AIA (15 in total) and find the timelag, $\tau_{AB}$, the temporal offset which maximizes the cross-correlation, in each pixel of the observed \AR{}. We consider all possible offsets over the interval $\pm6$ hours. Following the convention of \citet{viall_evidence_2012}, the channel pairs are ordered such that the ``hot'' channel is listed first, meaning that \textit{a positive timelag indicates cooling plasma}. The details of the cross-correlation and timelag calculations can be found in the appendix of \citetalias{barnes_understanding_2019}.

\begin{pycode}[manager_timelags]
fig = plt.figure(figsize=texfigure.figsize(
    pytex,
    scale=1 if is_onecolumn() else 2,
    height_ratio=2.95/5,
    figure_width_context='columnwidth'
))
plot_params = {
    'title': False, 
    'annotate': False,
    'vmin': -(5e3*u.s).to(u.s).value,
    'vmax': (5e3*u.s).to(u.s).value,
    'cmap': bgry_004_idl_cmap,
}
axes = []
for i,cp in enumerate(channel_pairs):
    m = Map(file_format.format('timelag', *cp))
    mc = Map(file_format.format('correlation', *cp))
    m = Map(m.data, m.meta, mask=np.where(mc.data<=correlation_threshold, True, False))
    m = m.submap(SkyCoord(Tx=-440*u.arcsec,Ty=-380*u.arcsec,frame=m.coordinate_frame),
                 SkyCoord(Tx=-185*u.arcsec,Ty=-125*u.arcsec,frame=m.coordinate_frame))
    ax = fig.add_subplot(3, 5, i+1, projection=m)
    axes.append(ax)
    im = m.plot(axes=ax, **plot_params)
    ax.grid(alpha=0)
    lon = ax.coords[0]
    lat = ax.coords[1]
    lon.set_ticks(number=3)
    lat.set_ticks(number=3,) 
    if i == 5:
        lat.set_ticklabel(rotation='vertical',)
        lat.set_axislabel(r'Helioprojective Latitude',)
    else:
        lat.set_ticklabel_visible(False)
    if i == 11:
        lon.set_axislabel(r'Helioprojective Longitude')
    else:
        lon.set_ticklabel_visible(False)
    xtext,ytext = m.world_to_pixel(SkyCoord(-190*u.arcsec, -360*u.arcsec, frame=m.coordinate_frame))
    ax.text(
        xtext.value,ytext.value,
        '{}-{} $\mathrm{{\AA}}$'.format(*cp),
        color='k',
        fontsize=plt.rcParams['legend.fontsize'],
        horizontalalignment='right',
        verticalalignment='bottom',
    )
plt.subplots_adjust(wspace=0.03, hspace=0.03)
cax = fig.add_axes([
    axes[0].get_position().get_points()[0,0],
    axes[4].get_position().get_points()[1,1] + 0.01,
    axes[-1].get_position().get_points()[1,0] - axes[0].get_position().get_points()[0,0], 
    0.02
])
cbar = fig.colorbar(im, cax=cax, orientation='horizontal')
cbar.ax.xaxis.set_ticks_position('top')
fig_timelags = manager_timelags.save_figure('timelags')
fig_timelags.caption = r'Timelag maps of \AR{} NOAA 1158 for all 15 channel pairs. The value of each pixel indicates the temporal offset, in seconds, which maximizes the cross-correlation \citepalias[see appendix of][]{barnes_understanding_2019}. The range of the colorbar is $\pm5000$ s. If $\max\mathcal{C}_{AB}<0.1$, the pixel is masked and colored white. Each map has been cropped to emphasize the core of the \AR{} such that the bottom left corner and top right corner of each image corresponds to $(-440\arcsec,-380\arcsec)$ and $(-185\arcsec,-125\arcsec)$, respectively.'
fig_timelags.figure_env_name = 'figure*'
fig_timelags.figure_width = r'\columnwidth' if is_onecolumn() else r'2\columnwidth'
fig_timelags.fig_str = fig_str
\end{pycode}
\py[manager_timelags]|fig_timelags|

\autoref{fig:timelags} shows the timelag maps of \AR{} NOAA 1158 for all 15 channel pairs. Blacks, blues, and greens indicate negative timelags while reds, oranges, and yellows correspond to positive timelags. Olive green denotes zero timelag. The range of the colorbar is $\pm5000$ s. If the maximum correlation in a given pixel is too small ($\max\mathcal{C}_{AB}<\py|correlation_threshold|$), the pixel is masked and colored white. Note that \citet{viall_survey_2017} carried out the timelag analysis on this same \AR{}, NOAA 1158. In this paper, we repeat this same analysis to ensure that we are treating the observed intensities in the exact same manner as the predicted intensities from \citetalias{barnes_understanding_2019} so that we can make meaningful comparisons between our observed and predicted diagnostics.

% Should we have more discussion on timelags here? This is really already covered in VK17...

\begin{pycode}[manager_timelags]
fig = plt.figure(figsize=texfigure.figsize(
    pytex,
    scale=1 if is_onecolumn() else 2,
    height_ratio=2.95/5,
    figure_width_context='columnwidth'
))
plot_params = {
    'title': False, 
    'annotate': False,
    'vmin': 0,
    'vmax': 1,
    'cmap': 'magma',
}
axes = []
for i,cp in enumerate(channel_pairs):
    m = Map(file_format.format('correlation', *cp))
    m = m.submap(SkyCoord(Tx=-440*u.arcsec,Ty=-380*u.arcsec,frame=m.coordinate_frame),
                 SkyCoord(Tx=-185*u.arcsec,Ty=-125*u.arcsec,frame=m.coordinate_frame))
    ax = fig.add_subplot(3, 5, i+1, projection=m)
    axes.append(ax)
    im = m.plot(axes=ax, **plot_params)
    ax.grid(alpha=0)
    lon = ax.coords[0]
    lat = ax.coords[1]
    lon.set_ticks(number=3)
    lat.set_ticks(number=3,) 
    if i == 5:
        lat.set_ticklabel(rotation='vertical',)
        lat.set_axislabel(r'Helioprojective Latitude',)
    else:
        lat.set_ticklabel_visible(False)
    if i == 11:
        lon.set_axislabel(r'Helioprojective Longitude')
    else:
        lon.set_ticklabel_visible(False)
    xtext,ytext = m.world_to_pixel(SkyCoord(-430*u.arcsec, -135*u.arcsec, frame=m.coordinate_frame))
    ax.text(
        xtext.value,ytext.value,
        '{}-{} $\mathrm{{\AA}}$'.format(*cp),
        color='w',
        fontsize=plt.rcParams['legend.fontsize'],
        horizontalalignment='left',
        verticalalignment='top',
    )
plt.subplots_adjust(wspace=0.03, hspace=0.03)
cax = fig.add_axes([
    axes[0].get_position().get_points()[0,0],
    axes[4].get_position().get_points()[1,1] + 0.01,
    axes[-1].get_position().get_points()[1,0] - axes[0].get_position().get_points()[0,0], 
    0.02
])
cbar = fig.colorbar(im, cax=cax, orientation='horizontal')
cbar.ax.xaxis.set_ticks_position('top')
fig_correlations = manager_timelags.save_figure('correlations')
fig_correlations.caption = r'Same as \autoref{fig:timelags} except here we show the maximum value of the cross-correlation, $\max\mathcal{C}_{AB}$, for each channel pair.'
fig_correlations.figure_env_name = 'figure*'
fig_correlations.figure_width = r'\columnwidth' if is_onecolumn() else r'2\columnwidth'
fig_correlations.fig_str = fig_str
\end{pycode}
\py[manager_timelags]|fig_correlations|

\autoref{fig:correlations} shows the maximum cross-correlation, $\max\mathcal{C}_{AB}$, in each pixel of the \AR{}. In this case, we do not mask pixels with $\max\mathcal{C}_{AB}<0.1$. Though the value of the cross-correlation can range from -1 (perfectly anti-correlated) to 1 (perfectly correlated), the colorbar only ranges from 0 to 1 as we are only interested in whether the lightcurves in each channel in the pair are in phase. 
%%%%%%%%%%%%%%%%%%%%%%%%%%%%%%%%%%%%%%%%%%%%%%%%%%%%%%%%%%%%%%%%%%%%%%%%%%%%%%%
%                                   Comparisons                               %
%%%%%%%%%%%%%%%%%%%%%%%%%%%%%%%%%%%%%%%%%%%%%%%%%%%%%%%%%%%%%%%%%%%%%%%%%%%%%%%
\section{Classification Model}\label{sec:compare}

\begin{pycode}[manager_ml]
manager_ml = texfigure.Manager(
    pytex, './',
    python_dir='python',
    fig_dir='figures',
    data_dir='data',
    number=4,
)
from classify import prep_data, classify_ar
X, Y, X_observation, bad_pixels = prep_data(
    manager_ml.data_dir,
    channel_pairs,
    heating,
    correlation_threshold=correlation_threshold,
    rsquared_threshold=rsquared_threshold,
    scale_slope=False,
    scale_timelag=False,
    scale_correlation=False,
)
# Dummy metadata for creating maps
meta = Map(os.path.join(manager_ml.data_dir, 'observations', 'timelag_171_131.fits')).meta

# ML classification code here
rf_options = {
    'n_estimators': 500,
    'max_features': 'sqrt',
    'criterion': 'gini',
    'max_depth': 30,
    'min_samples_leaf': 1,
    'min_samples_split': 2,
    'bootstrap': True,
    'oob_score': True,
    'max_leaf_nodes': None,
    'min_impurity_decrease': 0,
    'n_jobs': -1,
}
frequency_maps = {}
probability_maps = {}
test_error = {}

# EM slope only
f_map, p_maps, _, err = classify_ar(rf_options, X[:,-1:], Y, X_observation[:,-1:], bad_pixels,)
frequency_maps['a'] = f_map
probability_maps['a'] = p_maps
test_error['a'] = err

# Timelags, cross-correlation only
f_map, p_maps, _, err = classify_ar(rf_options, X[:,:-1], Y, X_observation[:,:-1], bad_pixels)
frequency_maps['b'] = f_map
probability_maps['b'] = p_maps
test_error['b'] = err

# EM slope, timelags, cross-correlation
f_map, p_maps, clf, err = classify_ar(rf_options, X, Y, X_observation, bad_pixels)
frequency_maps['c'] = f_map
probability_maps['c'] = p_maps
test_error['c'] = err

# Calculate feature importances
importances = clf.feature_importances_
i_important = np.argsort(importances)[::-1]
std = np.std([t.feature_importances_ for t in clf.estimators_], axis=0)

# Top 10 features
f_map, p_maps, _, err = classify_ar(rf_options, X[:, i_important[:10]], Y,
                                    X_observation[:, i_important[:10]], bad_pixels)
frequency_maps['d'] = f_map
probability_maps['d'] = p_maps
test_error['d'] = err
\end{pycode}

Rather than manually comparing our observations and simulations using all of the aformentioned diagnostics, we systematically assess our observations of NOAA 1158 in terms of the heating frequency by training a random forest classifier comprised of many decision tress on our predicted observables from \citetalias{barnes_understanding_2019}. We then use our trained model to classify each observed pixel in terms of high-, intermediate-, or low-frequency heating as defined in \autoref{eq:heating_types}. Unlike more traditional statistical methods, this approach allows us to simultaneously consider an arbitrarily large number of features when deciding which frequency best fits the observation. In the parlance of statistical learning, the heating frequency (low, intermediate, or high) is the \textit{class}, the emission measure slope, timelag, and cross-correlation are the \textit{features}, and the pixels are the \textit{samples}.

Following the explanation of \citet[chapter 8]{james_introduction_2013}, a decision tree recursively partitions the feature space of interest into a set of terminal nodes, or leaves, using a top-down, ``greedy'' approach called recursive binary splitting. At each node in the tree, a feature and an associated split point are chosen to maximize the number of observations of a single class in the resulting nodes. A common measure of the homogeneity or \textit{purity} of each node is the Gini index,
\begin{equation}\label{eq:gini-index}
    G_m = \sum_k \hat{p}_{mk} (1 - \hat{p}_{mk}),
\end{equation}
where $k$ indexes the class, $m$ indexes the node, and $\hat{p}_{mk}$ is the proportion of the observations at node $m$ that belong to class $k$. Note that as the purity of $m$ increases (i.e. $\hat{p}_{mk}\to0,1$), $G_m$ decreases ($G_m\to0$). Alternative measures of node purity may also be used \citep[see section 9.2.3 of][]{hastie_elements_2009}. For every resulting terminal node in the tree, the assigned class is determined by the most commonly occurring class of every observation at that node.

Decision trees are commonly used in classification problems because they are computationally efficient and relatively easy to interpret. Unlike many statistical learning techniques, decision trees do not assume any functional mapping between the inputs and outputs such that arbitrary, non-linear relatationships can be learned by the model. However, decision trees have two primary weaknesses: (1) they are known to have lower predictive accuracy than other more restrictive classification strategies and (2) they have high variance such that a single tree is not very robust to small changes in the training data \citep{james_introduction_2013}.

While individual decision trees are ``weak learners'', combined they give accurate and robust predictions. Random forest classifiers, first developed by \citet{breiman_random_2001}, provide an ensemble statistical learning method for combining many noisy, decorrelated decision trees in order to improve prediction accuracy and robustness. As in the bootstrap-aggregation, or ``bagging'', technique developed by \citet{breiman_bagging_1996}, each tree in the random forest is trained on only a subset of the total training data in order to reduce the variance of the model. Additionally, at each node in each tree, a random subset of the total features are considered as candidates for splitting in order to decrease the correlation between trees. A typical rule-of-thumb is to consider only $\left\lfloor\sqrt{p}\right\rfloor$ features at each split, where $p$ is the total number of features. This further reduces the variance and prevents a single feature from dominating the decision in every tree. Once each tree in the forest has been built using the training data, an unlabeled observation is classified by traversing each tree in the forest and taking the majority vote of the class at the terminal node of each tree. See chapter 15 of \citet{hastie_elements_2009} for a detailed discussion of random forests for both classification and regression.

\subsection{Data Preparation and Model Parameters}\label{sec:data-prep}

To build our classification model, we use the random forest classifier as implemented in the scikit-learn package for machine learning in Python \citep{pedregosa_scikit-learn_2011}. Using the predicted emission measure slopes, timelags, and maximum cross-correlations from \citetalias{barnes_understanding_2019}, we train a single random forest classifier composed of \py[manager_ml]|rf_options['n_estimators']| trees each with a maximum depth of \py[manager_ml]|rf_options['max_depth']|. At each node, $\left\lfloor\sqrt{31}\right\rfloor=\py|f'{int(np.sqrt(31)):.0f}'|$ possible split candidates are randomly selected from the $15\,\textup{timelags} + 15\,\textup{cross-correlations} + 1\,\textup{emission measure slope}=31$ total features. We note that all of these features are likely to be correlated with one another to some extent.

Before training the model, we flatten the predicted emission measure slope, timelag, and cross-correlation maps from \citetalias{barnes_understanding_2019} for the high-, intermediate-, and low-frequency heating cases into an array of length $n_xn_y$, where $n_x$ and $n_y$ are the dimensions of the predicted images. As before, we mask pixels where $r^2<\py|rsquared_threshold|$ for the emission measure slope fit and where $\max\mathcal{C}_{AB}<\py|correlation_threshold|$ for the cross-correlation. If a pixel is masked in one frequency case, we mask it in all other frequencies to ensure that we have an equal number of high-, intermediate-, and low-frequency data points. We stack each flattened array column-wise in features and row-wise in heating frequency such that all of the simulated data are encapsulated in a single data matrix $X$ of dimension $n\times p$. $p=31$ is the total number of features and $n=3n_xn_y - n_\textup{mask}=\py[manager_ml]|f'{X.shape[0]:.0f}'|$ is the total number of pixels for all heating frequencies minus those pixels which were masked in at least one feature of one frequency. The heating frequency label or class is numerically encoded as 0 (high), 1 (intermediate), or 2 (low) and similarly stacked to create a single response vector $Y$ of dimension $n\times1$. We apply a $2/3-1/3$ test-train split to $X$ and $Y$ such that approximately $1/3$ of the samples are reserved for model evaluation to ensure that our model has not overfit the data. This produces four separate matrices: $X_\textup{train},Y_\textup{train},X_\textup{test},Y_\textup{test}$. The data are not centered to a mean of 0 or scaled to unit standard deviation. By transforming the data in this manner, we are treating each pixel in the image as an independent sample with $p$ associated features per sample.

The same procedure as described above is applied to the observed emission measure slopes, timelags, and cross-correlations as shown in \autoref{fig:em-slopes}, \autoref{fig:timelags}, and \autoref{fig:correlations}, respectively. These results are flattened to a single data matrix $X^\prime$ of dimension $n^\prime\times p$, where $n^\prime=n_x^\prime n_y^\prime - n^\prime_\textup{mask}=\py[manager_ml]|f'{X_observation.shape[0]:.0f}'|$. The random forest model is trained on $X_\textup{train},Y_\textup{train}$ and model performance is evaluated on the ``unseen'' test set $X_\textup{test},Y_\textup{test}$. The trained model is then applied to $X^\prime$ in order to predict the heating frequency in each pixel, $Y^\prime$.

We do not apply any formal hyperparameter tuning or cross-validation procedure, though a manual exploration of the hyperparameters revealed that adding more than \py[manager_ml]|rf_options['n_estimators']| trees to the random forest provided only a marginal decrease in the test error while increasing the training time. Similarly, we find a maximum depth of \py[manager_ml]|rf_options['max_depth']| for each decision tree provides sufficient complexity to each tree as evaluated by the test error while not significantly increasing the computational cost of the training. However, in case A (see \autoref{tab:cases}), we find that less complex trees (i.e. lower maximum depth) result in a reduction in the misclassification error by $7-8\%$.

\subsection{Different Feature Combinations}\label{sec:feature-combos}

\begin{pycode}[manager_ml]
cases = ['a','b','c','d']
tab = {
    'Case': [c.upper() for c in cases],
    'Parameters': [r'$a$', r'$\tau_{AB},\mathcal{C}_{AB}$', r'$a,\tau_{AB},\mathcal{C}_{AB}$', r'Top 10 features from \autoref{tab:importance}'],
    r'$p$': [1, 30, 31, 10],
    'Error': [test_error[c] for c in cases],
    'High': [frequency_maps[c][frequency_maps[c] == 0].size/frequency_maps[c][~np.isnan(frequency_maps[c])].size for c in cases],
    'Inter.': [frequency_maps[c][frequency_maps[c] == 1].size/frequency_maps[c][~np.isnan(frequency_maps[c])].size for c in cases],
    'Low': [frequency_maps[c][frequency_maps[c] == 2].size/frequency_maps[c][~np.isnan(frequency_maps[c])].size for c in cases],
}
caption = r'The four different combinations of emission measure slope, timelag, and maximum cross-correlation. The third column lists the total number of features used in the classification. The fourth column gives the misclassification error as evaluated on $X_\textup{test},Y_\textup{test}$. The fifth, sixth, and seventh columns show the percentage of pixels labeled as high-, intermediate-, and low-frequency heating, respectively.\label{tab:cases}'
formats = {
    'Error': '%.2f',
    'High': '%.3f',
    'Inter.': '%.3f',
    'Low': '%.3f'
}
with io.StringIO() as f:
    ascii.write(tab, format='aastex', caption=caption, output=f, formats=formats, latexdict={ 'tabletype': r'deluxetable*'})
    table = f.getvalue()
\end{pycode}
\py[manager_ml]|table|

We apply the train-test-predict procedure described above to all four cases listed in \autoref{tab:cases}. In case A, the random forest classifier is trained only on the emission measure slope, $a$, such that the $X$ and $X^\prime$ have dimensions $n\times1$ and $n^\prime\times1$, respectively. In case B, the classifier is trained on the timelags and maximum cross-correlations for all 15 channel pairs for a total of $p=30$ features while in case C, every feature (emission measure slope, 15 timelags, 15 maximum cross-correlations) is used such that $p=31$. We discuss case D in \autoref{sec:feature-importance}. The fourth column in \autoref{tab:cases} lists the misclassification error as evaluated on the test set, $X_\textup{test},Y_\textup{test}$ and the fifth, sixth, and seventh columns show the fraction of pixels classified as high-, intermediate-, and low-frequency.

\begin{pycode}[manager_ml]
fig = plt.figure(figsize=texfigure.figsize(
    pytex,
    scale=1 if is_onecolumn() else 2,
    height_ratio=4/3*0.96,
    figure_width_context='columnwidth'
))
axes = []
for j,c in enumerate(('a','b','c','d')):
    for i,h in enumerate(heating):
        m = GenericMap(probability_maps[c][h], meta)
        m = m.submap(SkyCoord(Tx=-410*u.arcsec,Ty=-325*u.arcsec,frame=m.coordinate_frame),
                     SkyCoord(Tx=-225*u.arcsec,Ty=-150*u.arcsec,frame=m.coordinate_frame))
        ax = fig.add_subplot(4, 3, 3*j+i+1, projection=m)
        axes.append(ax)
        im = m.plot(axes=ax, annotate=False, title=False, vmin=0, vmax=1, cmap='viridis',)
        ax.grid(alpha=0)
        lon,lat = ax.coords
        lon.set_ticks(number=4)
        lat.set_ticks(number=2)
        if i == 0 and j==3:
            lon.set_axislabel('Helioprojective Longitude',)
            lat.set_axislabel('Helioprojective Latitude', )
            lat.set_ticklabel(rotation='vertical')
        else:
            lat.set_ticklabel_visible(False)
            lon.set_ticklabel_visible(False)
        if i == 0:
            xtext,ytext = m.world_to_pixel(SkyCoord(-400*u.arcsec,-165*u.arcsec,frame=m.coordinate_frame))
            ax.text(int(xtext.value), int(ytext.value), f'{c.capitalize()}', color='k', fontsize=plt.rcParams['legend.fontsize'])
        if j == 0:
            xtext,ytext = m.world_to_pixel(SkyCoord(-230*u.arcsec,-315*u.arcsec,frame=m.coordinate_frame))
            ax.text(int(xtext.value), int(ytext.value),
                    h.split('_')[0].capitalize(),
                    horizontalalignment='right',
                    verticalalignment='bottom',
                    color='k', fontsize=plt.rcParams['legend.fontsize'])
plt.subplots_adjust(wspace=0.03,hspace=0.03)
cax = fig.add_axes([
    axes[0].get_position().get_points()[0,0],
    axes[0].get_position().get_points()[1,1]+0.0075,
    axes[-1].get_position().get_points()[1,0] - axes[0].get_position().get_points()[0,0],
    0.01
])
cbar = fig.colorbar(im, cax=cax, orientation='horizontal')
cbar.ax.xaxis.set_ticks_position('top')
cbar.ax.tick_params(width=0.5)
cbar.outline.set_linewidth(0.5)
### Save ###
fig_probability_maps = manager_ml.save_figure('probability-maps')
fig_probability_maps.caption = r'Classification probability for each pixel in the observed \AR{}. The rows denote the different cases in \autoref{tab:cases} and the columns correspond to the different heating frequency classes. If any of the 31 features is not valid in a particular pixel, the pixel is masked and colored white. Note that summing over all heating probabilities in each row gives 1 in every pixel.'
fig_probability_maps.figure_env_name = 'figure*'
fig_probability_maps.figure_width = r'\columnwidth' if is_onecolumn() else r'2\columnwidth'
fig_probability_maps.fig_str = fig_str
\end{pycode}
\py[manager_ml]|fig_probability_maps|

After computing the predicted heating frequency for each $X^\prime$, the resulting classifications, $Y^\prime$, are mapped back to the corresponding observed pixel locations to create a map of the heating frequency. \autoref{fig:probability-maps} shows the probability that each pixel corresponds to a particular heating frequency. The rows denote the different feature subsets as given in \autoref{tab:cases} and the columns correspond to the different heating frequency classes. The class probability, as computed by the scikit-learn package, in each pixel is the mean class probability of all trees in the random forest classifier. The class probability for an individual tree is the proportion of all training samples at the terminal node that belong to that class.

\begin{pycode}[manager_ml]
fig = plt.figure(figsize=texfigure.figsize(
    pytex,
    scale=1 if is_onecolumn() else 2,
    height_ratio=0.96,
    figure_width_context='columnwidth'
))
axes = []
for i,c in enumerate(('a','b','c','d')):
    m = GenericMap(frequency_maps[c],meta)
    m = m.submap(SkyCoord(Tx=-410*u.arcsec,Ty=-325*u.arcsec,frame=m.coordinate_frame),
                 SkyCoord(Tx=-225*u.arcsec,Ty=-150*u.arcsec,frame=m.coordinate_frame))
    ax = fig.add_subplot(2, 2, i+1, projection=m)
    axes.append(ax)
    im = m.plot(
        axes=ax,
        title=False,
        annotate=False,
        vmin=-0.5,
        vmax=2.5,
        cmap='discrete_heating_frequency',
    )
    ax.grid(alpha=0)
    # Axes and ticks
    lon, lat = ax.coords
    if i == 2:
        lon.set_axislabel('Helioprojective Longitude',)
        lat.set_axislabel('Helioprojective Latitude',)
        lat.set_ticklabel(rotation='vertical')
    else:
        lon.set_ticklabel_visible(False)
        lat.set_ticklabel_visible(False)
    lon.set_ticks(number=4)
    lat.set_ticks(number=2)
    xtext,ytext = m.world_to_pixel(SkyCoord(-400*u.arcsec,-165*u.arcsec,frame=m.coordinate_frame))
    ax.text(int(xtext.value), int(ytext.value), f'{c.capitalize()}', color='k', fontsize=plt.rcParams['legend.fontsize'])
plt.subplots_adjust(wspace=0.03,hspace=0.03)
# Colorbar
cax = fig.add_axes([
    axes[0].get_position().get_points()[0,0],
    axes[0].get_position().get_points()[1,1] + 0.01,
    axes[-1].get_position().get_points()[1,0] - axes[0].get_position().get_points()[0,0],
    0.015
])
cbar = fig.colorbar(im, cax=cax,orientation='horizontal')
cbar.ax.xaxis.set_ticks_position('top')
cbar.set_ticks([-0.25,1,2.25])
cbar.ax.set_xticklabels([h.split('_')[0].capitalize() for h in heating],)
cbar.ax.tick_params(axis='x', which='both', length=0)
cbar.outline.set_linewidth(0.5)
# Save
fig_frequency_maps = manager_ml.save_figure('frequency-maps')
fig_frequency_maps.caption = r'Predicted heating frequency classification in each pixel of NOAA 1158 for each of the cases in \autoref{tab:cases}. The classification is determined by which heating frequency class has the highest mean probability over all trees in the random forest. Each pixel is colored blue, orange, or green depending on whether the most likely heating frequency is high, intermediate, or low, respectively. If any of the 31 features is not valid in a particular pixel, the pixel is masked and colored white.'
fig_frequency_maps.figure_env_name = 'figure*'
fig_frequency_maps.figure_width = r'\columnwidth' if is_onecolumn() else r'2\columnwidth'
fig_frequency_maps.fig_str = fig_str
\end{pycode}
\py[manager_ml]|fig_frequency_maps|

\autoref{fig:frequency-maps} shows the heating frequency, or class, as predicted by the random forest classifier in each pixel of the observed \AR{} for all four cases in \autoref{tab:cases}. The predicted class is the one which has the highest mean probability as computed over all trees in the random forest. Each pixel is colored blue, orange, or green depending on whether the class with the highest mean probability is high-, intermediate-, or low-frequency, respectively.

We find that for each combination of features in \autoref{tab:cases}, high-frequency heating dominates at the center of the \AR{}. This result is consistent with \AR{} core observations of hot, steady emission \citep{warren_evidence_2010,warren_constraints_2011}, steep emission measure slopes \citep[e.g.][]{winebarger_using_2011,del_zanna_evolution_2015}, and lack of variability in the intensity \citep[e.g.][]{antiochos_constraints_2003} and the velocity \citep{brooks_flows_2009} near the loop footpoints. The frequency classification map for case A is as expected given the observed emission measure slope map in the left panel of \autoref{fig:em-slopes} and the well-separated distributions of emission measure slopes from the different heating frequencies as shown in the right panel of \autoref{fig:em-slopes}.

From the fourth column of \autoref{tab:cases}, we find that adding more features to the classifier significantly improves the accuracy as computed on the test data set. However, comparing frequency maps in cases A ($p=1$) and C ($p=31$), we find that the general pattern of heating frequency across the \AR{} is similar despite the large differences between the misclassification error in case A (\py[manager_ml]|f"{tab['Error'][0]:.2f}"|) and case C (\py[manager_ml]|f"{tab['Error'][2]:.2f}"|). Additionally, looking at the seventh column of \autoref{tab:cases} and the panels in the third column of \autoref{fig:probability-maps}, we find that adding the timelag and maximum cross-correlation features significantly decreases the number of pixels classified as low frequency. Training the classifier on only the emission measure slope versus all of the features has a comparatively small impact on the fraction of pixels classified as intermediate frequency.

Interestingly, we find that when we use relatively shallow trees to build the random forest (e.g. a maximum depth of $<10$), the misclassification error on the test data set in case B becomes larger than that of case A, despite $p_\textup{B}>p_\textup{A}$. If very complex trees (maximum depth $>100$) are used in case A, the model overfits the data and the resulting classification becomes very noisy. However, in case B (and C), increasing the maximum depth continually decreases the test error, indicating that the model is not overfitting the data. This seems to suggest that the relationship between the heating frequency and the timelag, as well as the maximum cross-correlation, is much more complex than that of the relationship between the heating frequency and the emission measure slope.

\subsection{Feature Importance}\label{sec:feature-importance}

In addition to the predicted heating frequency, $Y^\prime$, for a set of features, $X^\prime$, it is also useful to know which of the $p$ features is most important in deciding to which class each observation (pixel) belongs. One measure of the importance of each feature is the decrease in the Gini index,
\begin{equation}\label{eq:gini_gain}
    \Delta G_m = \frac{M_m}{M}\left( G_m - \frac{M_{m,R}}{M_m}G_{m,R} - \frac{M_{m,L}}{M_m}G_{m,L} \right),
\end{equation}
where $G_m$ is the Gini index as given by \autoref{eq:gini-index}, $M$ is the total number of samples in the tree, $M_m$ is the total number of samples at parent node $m$, $M_{m,R(L)}$ is the total number of samples at the right (left) child node, and $G_{m,R(L)}$ is the Gini index at the right (left) child node \citep{sandri_bias_2008}. The importance of a particular feature in the random forest classification is then determined by summing \autoref{eq:gini_gain} over all nodes which split on that feature for every tree and averaging over all trees \citep{breiman_classification_1984}.

Note that if $G_{m,R}=G_{m,L}=G_m$, $\Delta G_m=0$ because the split at node $m$ did not improve the discrimination between classes compared to the split at the previous node. However, if the purity of the left or right node increases such that $G_{m,R}$ or $G_{m,L}$ decreases relative to $G_m$, $\Delta G_m > 0$ because the split at node $m$ has added information to the classifier by preferentially sorting samples of a single class to either the left or right child node.

\begin{pycode}[manager_ml]
all_labels = np.array([r'$\tau_{{{},{}}}$'.format(*cp) for cp in channel_pairs] +
                      [r'$\mathcal{{C}}_{{{},{}}}$'.format(*cp) for cp in channel_pairs] + [r'$a$'])
tab2 = {
    'Feature': all_labels[i_important[:10]],
    'Importance': importances[i_important[:10]]/importances[i_important[0]],
    r'$\sigma$': std[i_important[:10]],
}
caption = r'Ten most important features as determined by the random forest classifier in case C. The second column shows the variable importance as computed by \autoref{eq:gini_gain} and the third column, $\sigma$, is the standard deviation of the feature importance over all trees in the random forest. The second column is normalized such that the most important feature is equal to 1.\label{tab:importance}'
formats = { 'Importance': '%.4f', r'$\sigma$': '%.4f' }
with io.StringIO() as f:
    ascii.write(tab2, format='aastex', caption=caption, output=f, formats=formats, latexdict={'preamble': r'\tablewidth{\columnwidth}'})
    table = f.getvalue()
\end{pycode}
\py[manager_ml]|table|

\autoref{tab:importance} shows the ten most important features from case C as determined by \autoref{eq:gini_gain} summed over all nodes in each tree and averaged over all trees. The importance in the second column is normalized such that the most important feature is equal to 1. In case D as listed in the last row of \autoref{tab:cases}, only these ten features are used to train the random forest classifier and classify each observed pixel. The probability of each heating frequency for case D is shown in the last row of \autoref{fig:probability-maps} and the map of the most likely heating frequency in each pixel is shown in the bottom-right panel of \autoref{fig:frequency-maps}. We find that the probability maps in the last row of \autoref{fig:probability-maps} and the frequency map in the bottom right panel of \autoref{fig:frequency-maps} reveal approximately the same patterns of heating frequency across the \AR{} as the maps for case C in which all 31 features were included. Additionally, using less than $1/3$ of the total number of features, we achieve a misclassification error of \py[manager_ml]|f"{tab['Error'][3]:.2f}"|, comparable to case C.

According to \autoref{tab:importance}, the emission measure slope, $a$, has the most discriminating power in the random forest classifier. In particular, $a$ is more important than the second most important feature by over a factor of 2 and more important than the most important timelag feature by nearly an order of magnitude.

While useful, the feature importance in random forest classifiers should be interpreted cautiously, especially in cases where the features are correlated. The timelags, as well as the maximum cross-correlations, in all channel pairs are very strongly correlated. The emission measure slope is also likely correlated with the timelag and cross-correlation though perhaps more weakly so. In particular, \citet{altmann_permutation_2010} found that as the number of correlated features in a random forest classifier increased, the individual importance of each feature in the correlated group decreased and that for a very large number of correlated features ($\sim50$), the feature importance of each was close to zero. Here, we have at least two groups of 15 strongly correlated features each. Thus, the values shown in \autoref{tab:importance} for the timelag and cross-correlation should be regarded as lower bounds on the feature importance. However, the presence of highly-correlated or unimportant features is not expected to affect the robustness or accuracy of the classifier.

%%%%%%%%%%%%%%%%%%%%%%%%%%%%%%%%%%%%%%%%%%%%%%%%%%%%%%%%%%%%%%%%%%%%%%%%%%%%%%%
%                                   Discussion                                %
%%%%%%%%%%%%%%%%%%%%%%%%%%%%%%%%%%%%%%%%%%%%%%%%%%%%%%%%%%%%%%%%%%%%%%%%%%%%%%%
\section{Discussion}\label{sec:discussion}

% Wrap-up discussion of both papers
% What did we learn?
% agreement with past studies re hf heating in core (Warren 2011, Del Zanna 2015,)
% note that even single parameter rf with relatively high test error is useful as it provides a systematic way of assessing the data
% what does Table 2 of feature importances say about these variables?
% needs application to more ARs

%NOTE: WE CANNOT ASSESS THE ACCURACY OF THE WHOLE MODEL; WE CAN ONLY COMPARE DIFFERENT POINTS IN OUR PARAMETER SPACE

%NOTE: EFFECT OF FLATTENING TO A SINGLE ARRAY; HOW DOES THIS AFFECT DECISION MAKING? what consquences does this have (i.e. losing spatial information)
%% This goes here I think because it involves a discussion of the distribution of parameters for each heating frequency and how they might change


%%%%%%%%%%%%%%%%%%%%%%%%%%%%%%%%%%%%%%%%%%%%%%%%%%%%%%%%%%%%%%%%%%%%%%%%%%%%%%%
%                                   Summary and Conclusions                   %
%%%%%%%%%%%%%%%%%%%%%%%%%%%%%%%%%%%%%%%%%%%%%%%%%%%%%%%%%%%%%%%%%%%%%%%%%%%%%%%
\section{Conclusions and Summary}\label{sec:conclusions}

In this paper, the second in our series on constraining nanoflare heating properties, we have used predicted diagnostics from \citetalias{barnes_understanding_2019} to systematically classify each pixel of \AR{} NOAA 1158 in terms of frequency of energy deposition. In particular, we first collect 12 hours of full-resolution SDO/AIA observations of NOAA 1158 in six EUV channels: 94, 131, 171, 193, 211, and 335 \AA. We then co-align each image to a single time such that a given pixel in each image corresponds to approximately the same spatial coordinate and then crop the image to an area of $500\arcsec$-by-$500\arcsec$ centered on the \AR{}.

Next, we compute two diagnostics from these observed intensities: the emission measure slope and the time lag. We time-average the intensities of all six channels and use the method \citet{hannah_differential_2012} to compute the emission measure distribution in each pixel of the \AR{}. We then compute the emission measure slope, $a$, by fitting $\log_{10}\textup{EM}\sim a\log_{10}T$ over the temperature range $8\times10^5\,\textup{K}\le T < T_{peak}$. Next, we apply the time-lag analysis of \citet{viall_evidence_2012} to the full 12 hours of observations of NOAA 1158 and compute the time lag, $\tau_{AB}$, and maximum cross-correlation, $\max\mathcal{C}_{AB}$, in each pixel of the \AR{} for all possible ``hot-cool'' pairs of the six EUV channels, 15 in total.

Finally, we train a random forest classifier using the predicted emission measure slopes, time lags, and cross-correlations for three different heating frequencies from \citetalias{barnes_understanding_2019}. We then use our trained model to classify each observed pixel as consistent with either high-, intermediate-, or low-frequency heating and map the heating frequency across the entire \AR{}.

Our results can be summarized as follows:
\begin{enumerate}
    \item The distribution of observed emission measure slopes overlaps with the distributions of predicted emission measure slopes for high-, intermediate-, and low-frequency heating, suggesting a range of heating frequencies across the \AR{}.
    \item High-frequency heating dominates in the center of \AR{} and is coincident with the areas of large magnetic field strength.
    \item Intermediate-frequency heating is more likely in longer loops surrounding the center of the \AR{}. In most pixels, low-frequency heating, as defined in \autoref{eq:heating_types}, is not needed to explain the observed diagnostics
    \item The emission measure slope is the strongest predictor of the heating frequency. Radiative cooling and draining around $1-2$ MK as manifested in the maximum cross-correlation also appears to be a strong indicator relative to the time lags. However, the feature importance as determined by the classifier should be interpreted carefully.
\end{enumerate}

We have demonstrated an efficient and powerful technique for constraining the heating frequency in active region cores and, more broadly, for systematically comparing models and observations. Given that the diagnostics here are known to vary with age \citep[e.g.][]{schmelz_cold_2012,del_zanna_evolution_2015} and from one \AR{} to the next \citep{warren_systematic_2012,viall_survey_2017}, the next step is to apply this methodology to a large sample of \AR s to place strong constraints on the frequency of energy deposition in the magnetically-closed corona.

%%%%%%%%%%%%%%%%%%%%%%%%%%%%%%%%%%%%%%%%%%%%%%%%%%%%%%%%%%%%%%%%%%%%%%%%%%%%%%%
%                                   Acknowledgment                            %
%%%%%%%%%%%%%%%%%%%%%%%%%%%%%%%%%%%%%%%%%%%%%%%%%%%%%%%%%%%%%%%%%%%%%%%%%%%%%%%
\acknowledgments
%%%%%%%%%%%%%%%%%%%%%%%%%%%%%%%%%%%%%%%%%%%%%%%%%%%%%%%%%%%%%%%%%%%%%%%%%%%%%%%
%                                   Software                                  %
%%%%%%%%%%%%%%%%%%%%%%%%%%%%%%%%%%%%%%%%%%%%%%%%%%%%%%%%%%%%%%%%%%%%%%%%%%%%%%%
\software{
    Astropy\citep{the_astropy_collaboration_astropy_2018},
    Dask\citep{dask_development_team_dask_2016},
    Matplotlib\citep{hunter_matplotlib_2007},
    NumPy\citep{oliphant_guide_2006},
    scikit-learn\citep{pedregosa_scikit-learn_2011},
    seaborn\citep{waskom_seaborn_2018},
    SunPy\citep{sunpy_community_sunpypython_2015},
}
%%%%%%%%%%%%%%%%%%%%%%%%%%%%%%%%%%%%%%%%%%%%%%%%%%%%%%%%%%%%%%%%%%%%%%%%%%%%%%%
%                                   References                                %
%%%%%%%%%%%%%%%%%%%%%%%%%%%%%%%%%%%%%%%%%%%%%%%%%%%%%%%%%%%%%%%%%%%%%%%%%%%%%%%
\bibliographystyle{aasjournal.bst}
\bibliography{references.bib}
\end{document}
    