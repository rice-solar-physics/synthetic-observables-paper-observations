% spell-checker: disable %
%% Use AASTeX class, version 6.1
%% Allow for additional class options such as,
%%  twocolumn   : two text columns, 10 point font, single spaced article.
%%                This is the most compact and represent the final published
%%                derived PDF copy of the accepted manuscript from the publisher
%%  manuscript  : one text column, 12 point font, double spaced article.
%%  preprint    : one text column, 12 point font, single spaced article.  
%%  preprint2   : two text columns, 12 point font, single spaced article.
%%  modern      : a stylish, single text column, 12 point font, article with
%% 		            wider left and right margins. This uses the Daniel
%% 		            Foreman-Mackey and David Hogg design.
%%  astrosymb    : Loads Astrosymb font and define \astrocommands. 
%%  tighten      : Makes baselineskip slightly smaller, only works with 
%%                 the twocolumn substyle.
%%  times        : uses times font instead of the default
%%  linenumbers  : turn on lineno package.
%%  trackchanges : required to see the revision mark up and print its output
%%  longauthor   : Do not use the more compressed footnote style (default) for 
%%                 the author/collaboration/affiliations. Instead print all
%%                 affiliation information after each name. Creates a much
%%                 long author list but may be desirable for short author papers

\documentclass[modern]{aastex63}
%% Include packages
\usepackage{amsmath}
\usepackage{calc}
\include{pythontex}
%% Custom commands
\DeclareMathOperator*{\argmax}{argmax}
\newcommand{\AR}{active region}
\newcommand{\dem}{$\mathrm{EM}(T)$}
\newcommand{\twait}[1][]{t_{\textup{wait}#1}}
\renewcommand{\sectionautorefname}{Section}
\renewcommand{\subsectionautorefname}{Section}
\renewcommand{\subsubsectionautorefname}{Section}
%% Paper Aliases
\defcitealias{barnes_understanding_2019}{Paper I}
\defcitealias{viall_survey_2017}{VK17}
%%%%%%%%%%%%%%%%%%%%%%%%%%%%%%%%%%%%%%%%%%%%%%%%%%%%%%%%%%%%%%%%%%%%%%%%%%%%%%%
%                                   Body                                      %
%%%%%%%%%%%%%%%%%%%%%%%%%%%%%%%%%%%%%%%%%%%%%%%%%%%%%%%%%%%%%%%%%%%%%%%%%%%%%%%
\begin{document}

%%%%%%%%%%%%%%%%%%%%%%%%%%%%%%%%%%%%%%%%%%%%%%%%%%%%%%%%%%%%%%%%%%%%%%%%%%%%%%%
%                                   Title and Authors                         %
%%%%%%%%%%%%%%%%%%%%%%%%%%%%%%%%%%%%%%%%%%%%%%%%%%%%%%%%%%%%%%%%%%%%%%%%%%%%%%%
\title{Understanding Heating in Active Region Cores through Machine Learning II. Classifying Observations}
\author[0000-0001-9642-6089]{W. T. Barnes}
\affiliation{National Research Council Postdoctoral Research Associate residing at the Naval Research Laboratory, Washington, D.C. 20375}
\affiliation{Department of Physics \& Astronomy, Rice University, Houston, TX 77005-1827}
\author[0000-0002-3300-6041]{S. J. Bradshaw}
\affiliation{Department of Physics \& Astronomy, Rice University, Houston, TX 77005-1827}
\author[0000-0003-1692-1704]{N. M. Viall}
\affiliation{NASA Goddard Space Flight Center, Greenbelt, MD 20771}
\correspondingauthor{W. T. Barnes}
\email{will.barnes.ctr@nrl.navy.mil}

%%%%%%%%%%%%%%%%%%%%%%%%%%%%%%%%%%%%%%%%%%%%%%%%%%%%%%%%%%%%%%%%%%%%%%%%%%%%%%%
%                                   Abstract                                  %
%%%%%%%%%%%%%%%%%%%%%%%%%%%%%%%%%%%%%%%%%%%%%%%%%%%%%%%%%%%%%%%%%%%%%%%%%%%%%%%
\begin{abstract}
Constraining the frequency of energy deposition in magnetically-closed \AR{} cores requires sophisticated hydrodynamic simulations of the coronal plasma and detailed forward modeling of the optically-thin line-of-sight integrated emission.
However, understanding which set of model inputs best matches a set of observations is complicated by the need for any proposed heating model to simultaneously satisfy multiple observable constraints.
In this paper, we train a random forest classification model on a set of forward-modeled observable quantities, namely the emission measure slope, the peak temperature of the emission measure distribution, and the time lag and maximum cross-correlation between multiple pairs of AIA channels.
We then use our trained model to classify the heating frequency in every pixel of \AR{} NOAA 1158 using the observed emission measure slopes, peak temperatures, time lags, and maximum cross-correlations and are able to map the heating frequency across the entire active region.
We find that high-frequency heating dominates in the inner core of the \AR{} while intermediate frequency dominates closer to the periphery of the \AR{}.
Additionally, we assess the importance of each observed quantity in our trained classification model and find that the emission measure slope is the dominant feature in deciding with which heating frequency a given pixel is most consistent.
The technique presented here offers a very promising and widely applicable method for assessing observations in terms of detailed forward models given an arbitrary number of observable constraints.
\end{abstract}
%% Keywords
\keywords{Active solar corona (1988), Astronomy data analysis (1858), Solar extreme ultraviolet emission (1493), Random Forests (1935)}

%%%%%%%%%%%%%%%%%%%%%%%%%%%%%%%%%%%%%%%%%%%%%%%%%%%%%%%%%%%%%%%%%%%%%%%%%%%%%%%
%                                   Sections                                  %
%%%%%%%%%%%%%%%%%%%%%%%%%%%%%%%%%%%%%%%%%%%%%%%%%%%%%%%%%%%%%%%%%%%%%%%%%%%%%%%
%%%%%%%%%%%%%%%%%%%%%%%%%%%%%%%%%%%%%%%%%%%%%%%%%%%%%%%%%%%%%%%%%%%%%%%%%%%%%%%
%                                   Introduction                              %
%%%%%%%%%%%%%%%%%%%%%%%%%%%%%%%%%%%%%%%%%%%%%%%%%%%%%%%%%%%%%%%%%%%%%%%%%%%%%%%
\section{Introduction}\label{introduction}

% summarize paper 1
% emphasize need for detailed comparisons with observations
% two important points: systematic comparisons and multiple observables
% highlight past efforts to compare models and observations in context of heating
% highlight past ML work (e.g. Tajifirouze, Reale, et al.)
% note BV16 comparison between simulation and data
% note VK17 comparison with EM slopes from W12
% outline this paper

%%%%%%%%%%%%%%%%%%%%%%%%%%%%%%%%%%%%%%%%%%%%%%%%%%%%%%%%%%%%%%%%%%%%%%%%%%%%%%%
%                                   Data                                      %
%%%%%%%%%%%%%%%%%%%%%%%%%%%%%%%%%%%%%%%%%%%%%%%%%%%%%%%%%%%%%%%%%%%%%%%%%%%%%%%
\section{Observations and Analysis}\label{observations}

\begin{pycode}[manager_data]
manager_data = texfigure.Manager(
    pytex, './',
    python_dir='python',
    fig_dir='figures',
    data_dir='data',
    number=1,
)
\end{pycode}
\authorcomment1{Describe data collection and prep procedures; can give brief description of timelag and em procedures, mostly just pointing to Paper 1}

\begin{pycode}[manager_data]
fits_files = [
    'aia_lev1.5_20110212T153238_94_cutout.fits',
    'aia_lev1.5_20110212T153245_131_cutout.fits',
    'aia_lev1.5_20110212T153248_171_cutout.fits',
    'aia_lev1.5_20110212T153243_193_cutout.fits',
    'aia_lev1.5_20110212T153248_211_cutout.fits',
    'aia_lev1.5_20110212T153239_335_cutout.fits',
]
fig = plt.figure(figsize=texfigure.figsize(
    pytex,
    scale=1 if is_onecolumn() else 2,
    height_ratio=2/3,
    figure_width_context='columnwidth'
))
plt.subplots_adjust(hspace=0.0,wspace=0.03)
for i,f in enumerate(fits_files):
    m = Map(os.path.join(manager_data.data_dir, 'observations', f))
    m = m.submap(SkyCoord(Tx=-440*u.arcsec,Ty=-380*u.arcsec,frame=m.coordinate_frame),
                 SkyCoord(Tx=-185*u.arcsec,Ty=-125*u.arcsec,frame=m.coordinate_frame))
    ax = fig.add_subplot(2,3,i+1,projection=m)
    norm = ImageNormalize(vmin=0, vmax=m.data.max(), stretch=SqrtStretch())
    m.plot(axes=ax, title=False, annotate=False, norm=norm)
    ax.grid(alpha=0)
    lon,lat = ax.coords
    lon.set_ticks(color='k',number=4)
    lat.set_ticks(color='k',number=4)
    if i != 3:
        lon.set_ticklabel_visible(False)
        lat.set_ticklabel_visible(False)
    else:
        lat.set_ticklabel(rotation='vertical')
        lon.set_axislabel('Helioprojective Longitude')
        lat.set_axislabel('Helioprojective Latitude')
    xtext,ytext = m.world_to_pixel(SkyCoord(-420*u.arcsec, -150*u.arcsec, frame=m.coordinate_frame))
    ax.text(xtext.value, ytext.value, f'{m.meta["wavelnth"]} $\mathrm{{\AA}}$',
            color='w', fontsize=plt.rcParams['xtick.labelsize'])
fig_intensity_maps = manager_data.save_figure('intensity-maps')
fig_intensity_maps.caption = r'Maps of the observed intensity in all 6 EUV channels of AIA at a single point in time. The images of have been prepped, derotated, and cropped to \AR{} NOAA 1158.'
fig_intensity_maps.figure_env_name = 'figure*'
fig_intensity_maps.figure_width = r'\columnwidth' if is_onecolumn() else r'2\columnwidth'
fig_intensity_maps.fig_str = fig_str
\end{pycode}
\py[manager_data]|fig_intensity_maps|

%%%%%%%%%%%%%%%%%%%%%%%%%%%%%%%%%%%%%%%%%%%%%%%%%%%%%%%%%%%%%%%%%%%%%%%%%%%%%%%
%                                   EM Slopes                                 %
%%%%%%%%%%%%%%%%%%%%%%%%%%%%%%%%%%%%%%%%%%%%%%%%%%%%%%%%%%%%%%%%%%%%%%%%%%%%%%%
\subsection{Emission Measure Slopes}\label{em_slopes}

\begin{pycode}[manager_em]
manager_em = texfigure.Manager(
    pytex, './',
    python_dir='python',
    fig_dir='figures',
    data_dir='data',
    number=2,
)
from formatting import heating_palette
\end{pycode}

\authorcomment1{Describe EM results, what we did specifically to get them, show histograms of slopes as well as maps; maybe have some comparison with models here, i.e. lay histograms on top of each other}

\begin{pycode}[manager_em]
fig = plt.figure(figsize=texfigure.figsize(
    pytex,
    scale=1 if is_onecolumn() else 2,
    height_ratio=1/2,
    figure_width_context='columnwidth'
))
plt.subplots_adjust(wspace=0.31)
### Map ###
slope_map = Map(os.path.join(manager_em.data_dir, 'observations', 'em_slope.fits'))
slope_map = slope_map.submap(
    SkyCoord(Tx=-410*u.arcsec,Ty=-325*u.arcsec,frame=slope_map.coordinate_frame),
    SkyCoord(Tx=-225*u.arcsec,Ty=-150*u.arcsec,frame=slope_map.coordinate_frame))
cax = fig.add_axes([0.125, 0.82, 0.335, 0.025])
ax = fig.add_subplot(121, projection=slope_map)
im = slope_map.plot(
    axes=ax,
    cmap='viridis',
    vmin=2, vmax=5,
    title=False, annotate=False
)
ax.grid(alpha=0)
# HPC Axes
lon,lat = ax.coords
lon.set_ticks(number=4)
lat.set_ticks(number=2)
lat.set_ticklabel(rotation='vertical')
lon.set_axislabel('Helioprojective Longitude',)
lat.set_axislabel('Helioprojective Latitude',)
# Colorbar
cbar = fig.colorbar(im,cax=cax, orientation='horizontal',)
cbar.ax.xaxis.set_ticks_position('top')
cbar.set_ticks([2,3,4,5])
### Histograms ###
ax = fig.add_subplot(122,)
bins = np.arange(1,6,0.05)
colors = ['k'] + heating_palette()
# Plot Model EM Slopes
for i,h in enumerate(['observations'] + heating):
    d = Map(os.path.join(manager_em.data_dir, f'{h}', 'em_slope.fits')).data.flatten()
    h,b,_ = ax.hist(
        d[~np.isnan(d)],
        bins=bins,
        histtype='step',
        density=True,
        color=colors[i],
        label=h.split('_')[0].capitalize()
    )
# Ticks and Spines
ax.set_xlim(1.5,5.5);
ax.set_xticks([2,3,4,5])
ax.spines['top'].set_visible(False)
ax.spines['right'].set_visible(False)
ax.set_yticks(ax.get_yticks()[1:-1])
ax.spines['left'].set_bounds(ax.get_yticks()[0],ax.get_yticks()[-1])
ax.spines['bottom'].set_bounds(ax.get_xticks()[0],ax.get_xticks()[-1])
# Labels and legends
ax.set_xlabel(r'$a$')
ax.set_ylabel(r'Number of Pixels (Normalized)')
ax.legend(frameon=False,loc=1)
### Save ###
fig_em_slopes = manager_em.save_figure('em-slopes')
fig_em_slopes.caption = r'Left: Map of emission measure slope as computed from the time-averaged observed intensities. Right: Histogram of observed emission measure slope with model emission measure slopes overlaid.'
fig_em_slopes.figure_env_name = 'figure*'
fig_em_slopes.figure_width = r'\columnwidth' if is_onecolumn() else r'2\columnwidth'
fig_em_slopes.fig_str = fig_str
\end{pycode}
\py[manager_em]|fig_em_slopes|

%%%%%%%%%%%%%%%%%%%%%%%%%%%%%%%%%%%%%%%%%%%%%%%%%%%%%%%%%%%%%%%%%%%%%%%%%%%%%%%
%                                   Timelags                                  %
%%%%%%%%%%%%%%%%%%%%%%%%%%%%%%%%%%%%%%%%%%%%%%%%%%%%%%%%%%%%%%%%%%%%%%%%%%%%%%%
\subsection{Timelags}\label{timelags}

\begin{pycode}[manager_timelags]
manager_timelags = texfigure.Manager(
    pytex, './',
    python_dir='python',
    fig_dir='figures',
    data_dir='data',
    number=3,
)
file_format = os.path.join(manager_timelags.data_dir, 'observations', '{}_{}_{}.fits')
correlation_threshold = 0.1
\end{pycode}

\begin{pycode}[manager_timelags]
fig = plt.figure(figsize=texfigure.figsize(
    pytex,
    scale=1 if is_onecolumn() else 2,
    height_ratio=3/5,
    figure_width_context='columnwidth'
))
plot_params = {
    'title': False, 
    'annotate': False,
    'vmin': -(5e3*u.s).to(u.s).value,
    'vmax': (5e3*u.s).to(u.s).value,
    'cmap': 'RdYlBu_r',
}
cax = fig.add_axes([0.125, 0.89, 0.775, 0.02])
for i,cp in enumerate(channel_pairs):
    m = Map(file_format.format('timelag', *cp))
    mc = Map(file_format.format('correlation', *cp))
    m = Map(m.data, m.meta, mask=np.where(mc.data<=correlation_threshold, True, False))
    m = m.submap(SkyCoord(Tx=-440*u.arcsec,Ty=-380*u.arcsec,frame=m.coordinate_frame),
                 SkyCoord(Tx=-185*u.arcsec,Ty=-125*u.arcsec,frame=m.coordinate_frame))
    ax = fig.add_subplot(3, 5, i+1, projection=m)
    im = m.plot(axes=ax, **plot_params)
    ax.grid(alpha=0)
    lon = ax.coords[0]
    lat = ax.coords[1]
    lon.set_ticks(number=3)
    lat.set_ticks(number=3,) 
    if i == 5:
        lat.set_ticklabel(rotation='vertical',)
        lat.set_axislabel(r'Helioprojective Latitude',)
    else:
        lat.set_ticklabel_visible(False)
    if i == 11:
        lon.set_axislabel(r'Helioprojective Longitude')
    else:
        lon.set_ticklabel_visible(False)
    xtext,ytext = m.world_to_pixel(SkyCoord(-420*u.arcsec, -150*u.arcsec, frame=m.coordinate_frame))
    ax.text(
        xtext.value,ytext.value,
        '{}-{} $\mathrm{{\AA}}$'.format(*cp),
        color='k',
        fontsize=plt.rcParams['xtick.labelsize']
    )
plt.subplots_adjust(wspace=0.03, hspace=0.03)
cbar = fig.colorbar(im, cax=cax, orientation='horizontal')
cbar.ax.xaxis.set_ticks_position('top')
fig_timelags = manager_timelags.save_figure('timelags')
fig_timelags.caption = r'Timelag maps as calculated from intensity data from observations of \AR{} NOAA 1158 by SDO/AIA. Timelag maps are shown for every possible channel pair as indicated in the upper left corner of each map. The colorbar range from -5000 s to +5000 s.'
fig_timelags.figure_env_name = 'figure*'
fig_timelags.figure_width = r'\columnwidth' if is_onecolumn() else r'2\columnwidth'
fig_timelags.fig_str = fig_str
\end{pycode}
\py[manager_timelags]|fig_timelags|

\begin{pycode}[manager_timelags]
fig = plt.figure(figsize=texfigure.figsize(
    pytex,
    scale=1 if is_onecolumn() else 2,
    height_ratio=3/5,
    figure_width_context='columnwidth'
))
plot_params = {
    'title': False, 
    'annotate': False,
    'vmin': 0,
    'vmax': 1,
    'cmap': 'magma',
}
cax = fig.add_axes([0.125, 0.89, 0.775, 0.02])
for i,cp in enumerate(channel_pairs):
    m = Map(file_format.format('correlation', *cp))
    m = m.submap(SkyCoord(Tx=-440*u.arcsec,Ty=-380*u.arcsec,frame=m.coordinate_frame),
                 SkyCoord(Tx=-185*u.arcsec,Ty=-125*u.arcsec,frame=m.coordinate_frame))
    ax = fig.add_subplot(3, 5, i+1, projection=m)
    im = m.plot(axes=ax, **plot_params)
    ax.grid(alpha=0)
    lon = ax.coords[0]
    lat = ax.coords[1]
    lon.set_ticks(number=3)
    lat.set_ticks(number=3,) 
    if i == 5:
        lat.set_ticklabel(rotation='vertical',)
        lat.set_axislabel(r'Helioprojective Latitude',)
    else:
        lat.set_ticklabel_visible(False)
    if i == 11:
        lon.set_axislabel(r'Helioprojective Longitude')
    else:
        lon.set_ticklabel_visible(False)
    xtext,ytext = m.world_to_pixel(SkyCoord(-420*u.arcsec, -150*u.arcsec, frame=m.coordinate_frame))
    ax.text(
        xtext.value,ytext.value,
        '{}-{} $\mathrm{{\AA}}$'.format(*cp),
        color='w',
        fontsize=plt.rcParams['xtick.labelsize']
    )
plt.subplots_adjust(wspace=0.03, hspace=0.03)
cbar = fig.colorbar(im, cax=cax, orientation='horizontal')
cbar.ax.xaxis.set_ticks_position('top')
fig_correlations = manager_timelags.save_figure('correlations')
fig_correlations.caption = r'Same as \autoref{fig:timelags} except here we show the maximum value of the cross-correlation as derived from the observations.'
fig_correlations.figure_env_name = 'figure*'
fig_correlations.figure_width = r'\columnwidth' if is_onecolumn() else r'2\columnwidth'
fig_correlations.fig_str = fig_str
\end{pycode}
\py[manager_timelags]|fig_correlations|

\authorcomment2{Discussion of observed timelags and cross-correlation values}

\authorcomment2{Possibly put histograms of observations versus models here too}

\authorcomment1{Emphasize the point that no single heating model is consistent with the observations. Need multiple heating frequencies; this will lead into random forest stuff}
%%%%%%%%%%%%%%%%%%%%%%%%%%%%%%%%%%%%%%%%%%%%%%%%%%%%%%%%%%%%%%%%%%%%%%%%%%%%%%%
%                                   Comparisons                               %
%%%%%%%%%%%%%%%%%%%%%%%%%%%%%%%%%%%%%%%%%%%%%%%%%%%%%%%%%%%%%%%%%%%%%%%%%%%%%%%
\section{Classification Model}\label{compare}

\begin{pycode}[manager_ml]
manager_ml = texfigure.Manager(
    pytex, './',
    python_dir='python',
    fig_dir='figures',
    data_dir='data',
    number=4,
)
from formatting import heating_palette, heating_cmap
from classify import prep_data, classify_ar
X, Y, X_observation, bad_pixels = prep_data(
    manager_ml.data_dir,
    channel_pairs,
    heating,
    correlation_threshold=correlation_threshold,
    rsquared_threshold=rsquared_threshold,
    scale_slope=False,
    scale_timelag=True,
    scale_correlation=False,
)
# Dummy metadata for creating maps
meta = Map(os.path.join(manager_ml.data_dir, 'observations', 'timelag_171_131.fits')).meta
# ML classification code here
rf_options = {
    'n_estimators': 100,
    'max_features': 'sqrt',
    'criterion': 'gini',
    'max_depth': 25,
    'min_samples_leaf': 1,
    'bootstrap': True,
    'oob_score': True,
}
frequency_maps = {}
probability_maps = {}
test_error = {}
# EM slope only
f_map, p_maps, _, err = classify_ar(rf_options, X[:,-1:], Y, X_observation[:,-1:], bad_pixels,)
frequency_maps['a'] = f_map
probability_maps['a'] = p_maps
test_error['a'] = err
# Timelags, cross-correlation only
f_map, p_maps, _, err = classify_ar(rf_options, X[:,:-1], Y, X_observation[:,:-1], bad_pixels)
frequency_maps['b'] = f_map
probability_maps['b'] = p_maps
test_error['b'] = err
# EM slope, timelags, cross-correlation
f_map, p_maps, _, err = classify_ar(rf_options, X, Y, X_observation, bad_pixels)
frequency_maps['c'] = f_map
probability_maps['c'] = p_maps
test_error['c'] = err
\end{pycode}

\authorcomment1{Describe random forest technique; how is data prepared; what is the RF actually doing; Description should be very detailed}

\begin{pycode}[manager_ml]
cases = ['a','b','c']
tab = {
    'Name': [c.upper() for c in cases],
    'Parameters': [r'$a$', r'$\tau_{AB},\mathcal{C}_{AB}$', r'$a,\tau_{AB},\mathcal{C}_{AB}$'],
    'Test Error': [test_error[c] for c in cases],
    'High': [frequency_maps[c][frequency_maps[c] == 0].size/frequency_maps[c][~np.isnan(frequency_maps[c])].size for c in cases],
    'Intermediate': [frequency_maps[c][frequency_maps[c] == 1].size/frequency_maps[c][~np.isnan(frequency_maps[c])].size for c in cases],
    'Low': [frequency_maps[c][frequency_maps[c] == 2].size/frequency_maps[c][~np.isnan(frequency_maps[c])].size for c in cases],
}
caption = r'Three different parameter sets that we use to classify the heating frequency in the \AR{}.\label{tab:cases}'
formats = {'Test Error': '%.2f', 'High': '%.3f', 'Intermediate': '%.3f', 'Low': '%.3f'}
with io.StringIO() as f:
    ascii.write(tab, format='aastex', caption=caption, output=f, formats=formats)
    table = f.getvalue()
\end{pycode}
\py[manager_ml]|table|

\authorcomment1{Run classifier for EM, timelag+correlation, EM+timelag+correlation; compare results; show heating frequency maps, probability maps, and stack plots for all three cases}

\begin{pycode}[manager_ml]
fig = plt.figure(figsize=texfigure.figsize(
    pytex,
    scale=1 if is_onecolumn() else 2,
    height_ratio=0.96,
    figure_width_context='columnwidth'
))
axes = []
for j,c in enumerate(('a','b','c')):
    for i,h in enumerate(heating):
        m = GenericMap(probability_maps[c][h], meta)
        m = m.submap(SkyCoord(Tx=-410*u.arcsec,Ty=-325*u.arcsec,frame=m.coordinate_frame),
                     SkyCoord(Tx=-225*u.arcsec,Ty=-150*u.arcsec,frame=m.coordinate_frame))
        ax = fig.add_subplot(3, 3, 3*j+i+1, projection=m)
        axes.append(ax)
        im = m.plot(axes=ax, annotate=False, title=False, vmin=0, vmax=1, cmap='viridis',)
        ax.grid(alpha=0)
        lon,lat = ax.coords
        lon.set_ticks(number=4)
        lat.set_ticks(number=2)
        if i == 0 and j==2:
            lon.set_axislabel('Helioprojective Longitude',)
            lat.set_axislabel('Helioprojective Latitude', )
            lat.set_ticklabel(rotation='vertical')
        else:
            lat.set_ticklabel_visible(False)
            lon.set_ticklabel_visible(False)
        if i == 0:
            xtext,ytext = m.world_to_pixel(SkyCoord(-400*u.arcsec,-165*u.arcsec,frame=m.coordinate_frame))
            ax.text(int(xtext.value), int(ytext.value), f'{c.capitalize()}', color='k', fontsize=plt.rcParams['legend.fontsize'])
        if j == 0:
            xtext,ytext = m.world_to_pixel(SkyCoord(-235*u.arcsec,-315*u.arcsec,frame=m.coordinate_frame))
            ax.text(int(xtext.value), int(ytext.value),
                    h.split('_')[0].capitalize(),
                    horizontalalignment='right',
                    verticalalignment='bottom',
                    color='k', fontsize=plt.rcParams['legend.fontsize'])
plt.subplots_adjust(wspace=0.03,hspace=0.03)
cax = fig.add_axes([
    axes[0].get_position().get_points()[0,0],
    axes[0].get_position().get_points()[1,1]+0.0075,
    axes[-1].get_position().get_points()[1,0] - axes[0].get_position().get_points()[0,0],
    0.015
])
cbar = fig.colorbar(im, cax=cax, orientation='horizontal')
cbar.ax.xaxis.set_ticks_position('top')
### Save ###
fig_probability_maps = manager_ml.save_figure('probability-maps')
fig_probability_maps.caption = r'Probability maps for each case and each heating frequency'
fig_probability_maps.figure_env_name = 'figure*'
fig_probability_maps.figure_width = r'\columnwidth' if is_onecolumn() else r'2\columnwidth'
fig_probability_maps.fig_str = fig_str
\end{pycode}
\py[manager_ml]|fig_probability_maps|

\authorcomment1{Describe and interpret the results of the classification}

\begin{pycode}[manager_ml]
fig = plt.figure(figsize=texfigure.figsize(
    pytex,
    scale=1 if is_onecolumn() else 2,
    height_ratio=1/3,
    figure_width_context='columnwidth'
))
axes = []
for i,c in enumerate(('a','b','c')):
    m = GenericMap(frequency_maps[c],meta)
    m = m.submap(SkyCoord(Tx=-410*u.arcsec,Ty=-325*u.arcsec,frame=m.coordinate_frame),
                 SkyCoord(Tx=-225*u.arcsec,Ty=-150*u.arcsec,frame=m.coordinate_frame))
    ax = fig.add_subplot(1, 3, i+1, projection=m)
    axes.append(ax)
    im = m.plot(axes=ax, title=False,annotate=False, vmin=-0.5, vmax=2.5, cmap=heating_cmap())
    ax.grid(alpha=0)
    # Axes and ticks
    lon, lat = ax.coords
    if i == 0:
        lon.set_axislabel('Helioprojective Longitude',)
        lat.set_axislabel('Helioprojective Latitude',)
        lat.set_ticklabel(rotation='vertical')
    else:
        lon.set_ticklabel_visible(False)
        lat.set_ticklabel_visible(False)
    lon.set_ticks(number=4)
    lat.set_ticks(number=2)
    xtext,ytext = m.world_to_pixel(SkyCoord(-400*u.arcsec,-165*u.arcsec,frame=m.coordinate_frame))
    ax.text(int(xtext.value), int(ytext.value), f'{c.capitalize()}', color='k', fontsize=plt.rcParams['legend.fontsize'])
plt.subplots_adjust(wspace=0.03)
# Colorbar
cax = fig.add_axes([
    axes[0].get_position().get_points()[0,0],
    axes[0].get_position().get_points()[1,1] + 0.02,
    axes[-1].get_position().get_points()[1,0] - axes[0].get_position().get_points()[0,0],
    0.045
])
cbar = fig.colorbar(im, cax=cax,orientation='horizontal')
cbar.ax.xaxis.set_ticks_position('top')
cbar.set_ticks([-0.25,1,2.25])
cbar.ax.set_xticklabels([h.split('_')[0].capitalize() for h in heating],)
cbar.ax.tick_params(axis='x', which='both', length=0)
# Save
fig_frequency_maps = manager_ml.save_figure('frequency-maps')
fig_frequency_maps.caption = r'Heating frequency maps'
fig_frequency_maps.figure_env_name = 'figure*'
fig_frequency_maps.figure_width = r'\columnwidth' if is_onecolumn() else r'2\columnwidth'
fig_frequency_maps.fig_str = fig_str
\end{pycode}
\py[manager_ml]|fig_frequency_maps|

\authorcomment1{Table comparing three cases; list parameters used, test error, and percentage of pixels classified as each heating frequency}

%%%%%%%%%%%%%%%%%%%%%%%%%%%%%%%%%%%%%%%%%%%%%%%%%%%%%%%%%%%%%%%%%%%%%%%%%%%%%%%
%                                   Discussion                                %
%%%%%%%%%%%%%%%%%%%%%%%%%%%%%%%%%%%%%%%%%%%%%%%%%%%%%%%%%%%%%%%%%%%%%%%%%%%%%%%
\section{Discussion}\label{sec:discussion}

As evidenced in \autoref{fig:probability-maps} and \autoref{fig:frequency-maps}, we find that high-frequency heating is likely to dominate in the core of the \AR. 
Comparing the heating frequency maps in \autoref{fig:frequency-maps} for the different cases in \autoref{tab:cases}, this high-frequency classification seems largely due to the steep observed emission measure slopes in the center of the \AR{} as seen in \autoref{fig:em-slopes}.
This result is consistent with X-ray observations of hot, steadier emission \citep{warren_evidence_2010,warren_constraints_2011,winebarger_using_2011} as well as the result of \citet{del_zanna_evolution_2015} who found high values of the emission measure slope in the center of NOAA 11193.

Comparing case C in \autoref{fig:frequency-maps} with the observed magnetogram of NOAA 1158 shown in Figure 1 of \citetalias{barnes_understanding_2019}, we find that the areas of strongest magnetic field are spatially coincident with most of the pixels classified as high-frequency.
This suggests that those strands whose footpoints are rooted in areas of strong magnetic field strength are heated more frequently.
We will explore the relationship between the heating frequency and the underlying magnetic field strength in a future paper.

The longer loops surrounding the core are consistent with intermediate frequency heating.
Notably, the results from our classifier imply that low-frequency heating, as defined by \autoref{eq:heating_types}, is not needed to explain the observed time lags, suggesting that the waiting time on each strand in this \AR{} is likely to be on the order of or less than $\tau_\textup{cool}$.
This result is consistent with that of \citet{bradshaw_patterns_2016} who found that intermediate and high frequency nanoflares both produced time lags consistent with observations while their cooling experiment, similar to our low-frequency nanoflares, showed fundamental disagreements with the observed time-lag maps.

\explain{Note from Nicki: The patches of low frequency are a bit surprising.
I would have thought that it would have shown up as the individual loops, where we know that an instance of low heating occurred.
Maybe its because there are no locations where only low frequency occurs over the whole 12 hours (and low frequency for a short period is, in a sense, folded into the intermediate distribution already).
The patches where it is occurring - are those little loops? Or loop legs?}

\added{Additionally, we note that, in principle, one could model an entire active region with only steady steady heating and still reproduce the distribution of observed emission measure slopes.
For example, the observed shallow slopes on periphery could be consistent with steady 1MK, 2 MK, and 3 MK loops all emitting along the LOS.
Similarly, the steep slopes near the inner core are consistent with only steady 3 MK loops along the LOS.
This is also the case with $T_{peak}$
However, it has been exhaustively shown that truly steady heating, in which the energy deposition is constant in time, is not consistent with observed time lags or cross-correlation values \citep[e.g.][]{viall_signatures_2016}.
Thus, we do not explicitly test a steady heating model here and note that even our high-frequency heating model produces variability in the observed emission.}

After the emission measure slope, $a$, the next three most important features in the classification are the maximum cross correlations for the 211-193, 193-171, and 211-171 \AA{} channel pairs.
These three channels, 211 \AA{}, 193 \AA{}, and 171 \AA{}, peak sequentially in temperature at 1.8, 1.6, and 0.8 MK, respectively (see \autoref{fig:aia-response}), suggesting that the plasma dynamics in this temperature range, which are dominated by radiative cooling and draining \citep[e.g.][]{bradshaw_cooling_2005,bradshaw_cooling_2010,bradshaw_new_2010}, are coupled to, and indicative of, the frequency at which energy is deposited in the plasma and that thermal conduction has not erased all signatures of the heating.
A strand heated by low-frequency nanoflares will be allowed to cool well below 1 MK, producing a strong cross-correlation in these channel pairs, while a strand heated by high-frequency nanoflares will rarely be allowed to cool below the equilibrium temperature such that the cross-correlation, particularly in the 171 \AA{} channel pairs, is likely to be relatively low.
This cooling behavior is illustrated for a single strand in Figure 3 of \citetalias{barnes_understanding_2019}.

\added{Interestingly, we note that there are no channel pairs, either for the time lag or maximum cross-correlation, that include the 94 \AA{} channel in the ten most important features as shown in \autoref{tab:importance}.    
Observed time lags \citep{viall_evidence_2012,viall_survey_2017} show a transition between being dominated by the ``hot'' 94 \AA{} emission in the inner core to ``cool'' 94 \AA{}  emission in the periphery as evidenced by time lags changing from positive to negative, respectively.
Two proposed explanations for this switchover are that either impulsive heating in the cores is more energetic or it is more frequent.
The inability of the 94 \AA{} pairs to effectively discriminate between heating frequencies, as measured by the feature importance, points to the switchover being dominated by the energy rather than the frequency.
This is also confirmed by the top row of Figure 8 of \citetalias{barnes_understanding_2019} which shows the positive-negative switchover between the inner core and the periphery for all heating frequencies.}

While the maximum cross-correlation in the 211-193 \AA{} channel pair (see bottom row of \autoref{fig:correlations}) is very high across the whole \AR{}, the 193-171 \AA{} and 211-171 \AA{} maps (as well as the other 171 \AA{} pairs except for 171-131 \AA{}) show a comparatively low cross-correlation.
Combined with the heating frequency maps in \autoref{fig:frequency-maps} which indicate that the center of the \AR{} is consistent with high-frequency heating, this suggests that many of the loops in the core are kept from cooling much below 1.6 MK.

\added{Note from Nicki: The 171-131 channel turned out to be an enormous constraint on low frequency heating.
There is cooling through almost all of the channels on a regular basis - all except 131.
The timelag maps showing no cooling into but almost never through the 131 bandpass really was the nail in the coffin for exclusively low frequency heating.
It was the Bradshaw and Viall paper where we first realized how important that observation is.
This is why I said to point out earlier in the paper that the 211-193 has small, but non-zero timelags.}

An important caveat to this method for systematic comparison as we have applied it here is that the random forest classifier trained on the simulated emission measure slopes, peak temperatures, time lags, and maximum cross-correlations cannot provide any assessment of the accuracy of our model as described in \citetalias{barnes_understanding_2019}.
The classifier can only say, out of the provided classes (high-, intermediate-, or low-frequency), which type of heating \textit{best} describes the data.
However, given another method for assessing the heating frequency or perhaps some alternative forward-modeling approach, a random forest classifier could be used to compare these two methods.
In this way, machine learning also provides a promising strategy for reconciling different modeling approaches.

%%%%%%%%%%%%%%%%%%%%%%%%%%%%%%%%%%%%%%%%%%%%%%%%%%%%%%%%%%%%%%%%%%%%%%%%%%%%%%%
%                                   Summary and Conclusions                   %
%%%%%%%%%%%%%%%%%%%%%%%%%%%%%%%%%%%%%%%%%%%%%%%%%%%%%%%%%%%%%%%%%%%%%%%%%%%%%%%
\section{Conclusions and Summary}\label{sec:conclusions}

%% Paper 1 summary (1 paragraph!)

\added{In \citetalias{barnes_understanding_2019}, we carried out a series of numerical simulations to understand how the frequency of energy deposition is manifested in observable signatures in quiescent active regions.
By combining potential field extrapolations, efficient hydrodynamic modeling, and our novel and efficient forward modeling pipeline, we produced AIA images of \AR{} NOAA 1158 for all six EUV channels for $\approx8$ h of simulation time for high-, intermediate-, and low-frequency heating.
From these simulated intensities, we computed the emission measure slope and the time lag for all possible AIA channel pairs in each pixel of the \AR{} for all heating frequencies.
We found that the emission measure slope becomes increasingly shallow as heating frequency decreases, but as the heating frequency increases, the distribution of slopes peaks at higher values and becomes more broad.
Additionally, as the heating frequency decreased, the spatial distribution of time lags was increasingly determined by the distribution of loop lengths over the \AR{}.
Importantly, we also found that negative time lags in channel pairs where the second channel is 131 \AA{} provide a possible diagnostic for $\ge10$ MK plasma.}

In this paper, the second in our series on constraining nanoflare heating properties, we have used predicted diagnostics from \citetalias{barnes_understanding_2019} to systematically classify each pixel of \AR{} NOAA 1158 in terms of frequency of energy deposition.
In particular, we first collect 12 h of full-resolution SDO/AIA observations of NOAA 1158 in six EUV channels: 94, 131, 171, 193, 211, and 335 \AA.
We then co-align each image to a single time such that a given pixel in each image corresponds to approximately the same spatial coordinate and then crop the image to an area of $500\arcsec$-by-$500\arcsec$ centered on the \AR{}.

Next, we time-average the intensities of all six channels and use the method \citet{hannah_differential_2012} to compute the emission measure distribution in each pixel of the \AR{}.
We compute the peak temperature of the emission measure distribution, $T_{peak}$, as well as the emission measure slope, $a$, by fitting $\log_{10}\textup{EM}\sim a\log_{10}T$ over the temperature range $8\times10^5\,\textup{K}\le T < T_{peak}$. 
Additionally, we apply the time-lag analysis of \citet{viall_evidence_2012} to the full 12 h of observations of NOAA 1158 and compute the time lag, $\tau_{AB}$, and maximum cross-correlation, $\max\mathcal{C}_{AB}$, in each pixel of the \AR{} for all possible pairs of the six EUV channels, 15 in total.

Finally, we train a random forest classifier using the predicted emission measure slopes, peak temperatures, time lags, and cross-correlations for three different heating frequencies from \citetalias{barnes_understanding_2019}.
We then use our trained model to classify each observed pixel as consistent with either high-, intermediate-, or low-frequency heating (where the frequency is parameterized relative to the loop cooling time) and map the heating frequency across the entire \AR{}.

Our results can be summarized as follows:
\begin{enumerate}
    \item The distribution of observed emission measure slopes overlaps with the distributions of predicted emission measure slopes for high-, intermediate-, and low-frequency heating, suggesting a range of heating frequencies across the \AR{}.
    \item High-frequency heating dominates in the center of \AR{} and is coincident with loops whose footpoints are rooted in strong magnetic field.
    \item Intermediate-frequency heating is more likely in longer strands surrounding the center of the \AR{}. In most pixels, low-frequency heating, as defined in \autoref{eq:heating_types}, is not needed to explain the observed diagnostics.
    \item The emission measure slope is the strongest single-measure predictor of the heating frequency. Radiative cooling and draining around $1-2$ MK as manifested in the maximum cross-correlation also appears to be a strong indicator relative to the time lags. However, the feature importance as determined by the classifier should be interpreted carefully.
\end{enumerate}

We have demonstrated an efficient and powerful technique for constraining the heating frequency in active region cores and, more broadly, for systematically comparing models and observations.
While we have applied this technique for a particular set of heating parameters and a particular forward model of a single \AR{}, we emphasize that this approach for comparing models and observations is broadly applicable to any set of heating inputs and forward modeling technique. 
Given that the diagnostics here are known to vary with age \citep[e.g.][]{schmelz_cold_2012,del_zanna_evolution_2015} and from one \AR{} to the next \citep{warren_systematic_2012,viall_survey_2017}, the next step is to apply this methodology to a large sample of \AR s to place strong constraints on the frequency of energy deposition in the magnetically-closed corona.


%%%%%%%%%%%%%%%%%%%%%%%%%%%%%%%%%%%%%%%%%%%%%%%%%%%%%%%%%%%%%%%%%%%%%%%%%%%%%%%
%                                   Acknowledgment                            %
%%%%%%%%%%%%%%%%%%%%%%%%%%%%%%%%%%%%%%%%%%%%%%%%%%%%%%%%%%%%%%%%%%%%%%%%%%%%%%%
\acknowledgments
This research makes use of \added{version 0.9.5 \citep{stuart_mumford_2018_2155946}} of sunpy, an open-source and free community-developed solar data analysis package written in Python \citep{the_sunpy_community_sunpy_2020}.
\added{We use v0.9.5 in this work to maintain consistency with the software environment used in \citetalias{barnes_understanding_2019} which was completed prior to the v1.0 release of sunpy.}
SJB and WTB were supported by the NSF through CAREER award AGS-1450230.
WTB was supported by NASA’s \textit{Hinode} program.
\textit{Hinode} is a Japanese mission developed and launched by ISAS/JAXA with NAOJ as a domestic partner and NASA and STFC (UK) as international partners.
It is operated by these agencies in cooperation with ESA and NSC (Norway).
The work of NMV was supported by the NASA Supporting Research program.
The complete source of this paper, including the data, code, and instructions for training the classification model, can be found at \href{https://github.com/rice-solar-physics/synthetic-observables-paper-observations}{github.com/rice-solar-physics/synthetic-observables-paper-observations}.

%%%%%%%%%%%%%%%%%%%%%%%%%%%%%%%%%%%%%%%%%%%%%%%%%%%%%%%%%%%%%%%%%%%%%%%%%%%%%%%
%                                   Facilities                                %
%%%%%%%%%%%%%%%%%%%%%%%%%%%%%%%%%%%%%%%%%%%%%%%%%%%%%%%%%%%%%%%%%%%%%%%%%%%%%%%
\facility{SDO(AIA)}

%%%%%%%%%%%%%%%%%%%%%%%%%%%%%%%%%%%%%%%%%%%%%%%%%%%%%%%%%%%%%%%%%%%%%%%%%%%%%%%
%                                   Software                                  %
%%%%%%%%%%%%%%%%%%%%%%%%%%%%%%%%%%%%%%%%%%%%%%%%%%%%%%%%%%%%%%%%%%%%%%%%%%%%%%%
\software{
    astropy \citep[v3.1.0,][]{the_astropy_collaboration_astropy_2018,the_astropy_collaboration_2018_4080996},
	dask \citep[v1.0.0,][]{rocklin_dask:_2015},
	drms \citep[v0.5,][]{glogowski_drms_2019,kolja_glogowski_2019_2572850},
    matplotlib \citep[v3.0.2,][]{hunter_matplotlib_2007,thomas_a_caswell_2018_1482099},
	numpy \citep[v1.15.4,][]{harris_array_2020},
	PythonTeX \citep[v0.16,][]{poore_pythontex_2015},
    scikit-learn \citep[v0.20,][]{pedregosa_scikit-learn_2011,olivier_grisel_2019_2582066},
	seaborn \citep[v0.9.0,][]{michael_waskom_2018_1313201},
	scipy \citep[v1.1.0,][]{virtanen_scipy_2020, pauli_virtanen_2018_1241501},
	SolarSoftware \citep{freeland_data_1998},
    sunpy \citep[v0.9.5,][]{stuart_mumford_2018_2155946}
}

%%%%%%%%%%%%%%%%%%%%%%%%%%%%%%%%%%%%%%%%%%%%%%%%%%%%%%%%%%%%%%%%%%%%%%%%%%%%%%%
%                                   References                                %
%%%%%%%%%%%%%%%%%%%%%%%%%%%%%%%%%%%%%%%%%%%%%%%%%%%%%%%%%%%%%%%%%%%%%%%%%%%%%%%
\bibliographystyle{aasjournal.bst}
\bibliography{references.bib,software.bib}

\listofchanges

\end{document}
    