%%%%%%%%%%%%%%%%%%%%%%%%%%%%%%%%%%%%%%%%%%%%%%%%%%%%%%%%%%%%%%%%%%%%%%%%%%%%%%%
%                                   Discussion                                %
%%%%%%%%%%%%%%%%%%%%%%%%%%%%%%%%%%%%%%%%%%%%%%%%%%%%%%%%%%%%%%%%%%%%%%%%%%%%%%%
\section{Discussion}\label{sec:discussion}

% Discuss classification results in light of EM slopes and qualitative assessment
% Wrap-up discussion of both papers
% What did we learn?
% mention cross-validation?
% agreement with past studies re hf heating in core (Warren 2011, Del Zanna 2015,)
% note that even single parameter rf with relatively high test error is useful as it provides a systematic way of assessing the data
% what does Table 2 of feature importances say about these variables?
% needs application to more ARs

%NOTE: WE CANNOT ASSESS THE ACCURACY OF THE WHOLE MODEL; WE CAN ONLY COMPARE DIFFERENT POINTS IN OUR PARAMETER SPACE


\subsection{Comparing Emission Measure Slopes and Timelags}

\subsection{Heating Frequency Classification}
