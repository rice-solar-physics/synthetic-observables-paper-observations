%%%%%%%%%%%%%%%%%%%%%%%%%%%%%%%%%%%%%%%%%%%%%%%%%%%%%%%%%%%%%%%%%%%%%%%%%%%%%%%
%                                   Discussion                                %
%%%%%%%%%%%%%%%%%%%%%%%%%%%%%%%%%%%%%%%%%%%%%%%%%%%%%%%%%%%%%%%%%%%%%%%%%%%%%%%
\section{Discussion}\label{sec:discussion}

As evidenced in \autoref{fig:probability-maps} and \autoref{fig:frequency-maps}, we find that high-frequency heating is likely to dominate in the core of the \AR. Comparing the heating frequency maps in \autoref{fig:frequency-maps} for the different cases in \autoref{tab:cases}, this high-frequency classification seems largely due to the steep observed emission measure slopes in the center of the \AR{} as seen in \autoref{fig:em-slopes}. This result is consistent with X-ray observations of hot, steady emission \citep{warren_evidence_2010,warren_constraints_2011,winebarger_using_2011} as well as the result of \citet{del_zanna_evolution_2015} who found high values of the emission measure slope in the center of NOAA 11193.

Comparing case C in \autoref{fig:frequency-maps} with the observed magnetogram of NOAA 1158 shown in Figure 1 of \citetalias{barnes_understanding_2019}, we find that the areas of strongest magnetic field are spatially coincident with most of the pixels classified as high-frequency. This suggests that those strands whose footpoints are rooted in areas of large magnetic field strength are heated more frequently. We will explore the relationship between the heating frequency and the underlying magnetic field strength in a future paper.

The longer loops surrounding the core are consistent with intermediate frequency heating. Notably, the results from our classifier imply that low-frequency heating, as defined by \autoref{eq:heating_types}, is not needed to explain the observed time lags, suggesting that the waiting time on each strand in this \AR{} is likely to be on the order of or less than $\tau_\textup{cool}$. This result is consistent with that of \citet{bradshaw_patterns_2016} who found that intermediate and high frequency nanoflares both produced time lags consistent with observations while their cooling experiment, similar to our low-frequency nanoflares, showed fundamental disagreements with the observed time-lag maps.

After the emission measure slope, $a$, the next three most important features in the classification are the maximum cross correlations for the 211-193, 193-171, and 211-171 \AA{} channel pairs. These three channels, 211 \AA{}, 193 \AA{}, and 171 \AA{}, peak sequentially in temperature at 1.8, 1.6, and 0.8 MK, respectively (see \autoref{fig:aia-response}), suggesting that the plasma dynamics in this temperature range, which are dominated by radiative cooling and draining \citep[e.g.][]{bradshaw_cooling_2005,bradshaw_cooling_2010,bradshaw_new_2010}, are coupled to, and indicative of, the frequency at which energy is deposited in the plasma and that thermal conduction has not erased all signatures of the heating. A strand heated by low-frequency nanoflares will be allowed to cool well below 1 MK, producing a strong cross-correlation in these channel pairs, while a strand heated by high-frequency nanoflares will rarely be allowed to cool below the equilibrium temperature such that the cross-correlation, particularly in the 171 \AA{} channel pairs, is likely to be relatively low. This cooling behavior is illustrated for a single strand in Figure 3 of \citetalias{barnes_understanding_2019}.

While the maximum cross-correlation in the 211-193 \AA{} channel pair (see bottom row of \autoref{fig:correlations}) is very high across the whole \AR{}, the 193-171 \AA{} and 211-171 \AA{} maps (as well as the other 171 \AA{} pairs except for 171-131 \AA{}) show a comparatively low cross-correlation. Combined with the heating frequency maps in \autoref{fig:frequency-maps} which indicate that the center of the \AR{} is consistent with high-frequency heating, this suggests that many of the loops in the core are kept from cooling much below 1.6 MK.

An important caveat to this method for systematic comparison as we have applied it here is that the random forest classifier trained on the simulated emission measure slopes, time lags, and maximum cross-correlations cannot provide any assessment of the accuracy of our model as described in \citetalias{barnes_understanding_2019}. The classifier can only say, out of the provided classes (high-, intermediate-, or low-frequency), which type of heating best describes the data. However, given another method for assessing the heating frequency or perhaps some alternative forward-modeling approach, a random forest classifier could be used to compare these two methods. In this way, machine learning also provides a promising strategy for reconciling different modeling approaches.
