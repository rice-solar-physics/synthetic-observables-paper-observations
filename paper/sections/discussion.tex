%%%%%%%%%%%%%%%%%%%%%%%%%%%%%%%%%%%%%%%%%%%%%%%%%%%%%%%%%%%%%%%%%%%%%%%%%%%%%%%
%                                   Discussion                                %
%%%%%%%%%%%%%%%%%%%%%%%%%%%%%%%%%%%%%%%%%%%%%%%%%%%%%%%%%%%%%%%%%%%%%%%%%%%%%%%
\section{Discussion}\label{sec:discussion}

As evidenced in \autoref{fig:probability-maps} and \autoref{fig:frequency-maps}, we find that high-frequency heating is likely to dominate in the core of the \AR. 
Comparing the heating frequency maps in \autoref{fig:frequency-maps} for the different cases in \autoref{tab:cases}, this high-frequency classification seems largely due to the steep observed emission measure slopes in the center of the \AR{} as seen in \autoref{fig:em-slopes}.
\added{Our heating classification is consistent with the results of \citet{warren_systematic_2012} who computed \dem{} for a single pixel near the periphery of \AR{} NOAA 1158 and found $a=2.7\pm0.32$.
This is consistent with our calculation of $a$ in this same region (see left panel of \autoref{fig:em-slopes}).
While \citet{warren_systematic_2012} cite this as evidence of ``high-frequency'' heating, we note that their definition of ``high-frequency'' overlaps with our classifcation of both high- and intermediate-frequency heating such that their conclusions are consistent with the classification results shown in \autoref{fig:frequency-maps}.}
Additionally, this result is consistent with X-ray observations of hot, steadier emission \citep{warren_evidence_2010,warren_constraints_2011,winebarger_using_2011} as well as the result of \citet{del_zanna_evolution_2015} who found high values of the emission measure slope in the center of NOAA 11193.

Comparing case C in \autoref{fig:frequency-maps} with the observed magnetogram of NOAA 1158 shown in Figure 1 of \citetalias{barnes_understanding_2019}, we find that areas of stronger magnetic field are spatially coincident with most of the pixels classified as high-frequency.
This suggests that those strands whose footpoints are rooted in areas of strong magnetic field strength are heated more frequently.
We will explore the relationship between the heating frequency and the underlying magnetic field strength in a future paper.

The longer loops surrounding the core are consistent with intermediate frequency heating.
Notably, the results from our classifier imply that low-frequency heating, as defined by \autoref{eq:heating_types}, is not needed to explain the observed time lags, suggesting that the waiting time on each strand in this \AR{} is likely to be on the order of or less than $\tau_\textup{cool}$.
This result is consistent with that of \citet{bradshaw_patterns_2016} who found that intermediate and high frequency nanoflares both produced time lags consistent with observations while their cooling experiment, similar to our low-frequency nanoflares, showed fundamental disagreements with the observed time-lag maps.

Unlike the high- and intermediate-frequency cases, we find that the groups of pixels most consistent with low-frequency heating do not appear as spatially-coherent loop-like structures, but instead have a ``patchy'' appearance.
This is especially true in cases C and D.
In particular, in the bottom row of \autoref{fig:frequency-maps}, these low-frequency patches appear to be near the footpoints of longer loops in the \AR{}.
We hypothesize that this lack of spatial coherence in the identification of low-frequency heating could be due to multiple overlapping structures, consistent with higher-frequency heating, along the LOS.
Additionally, these structures may also be undergoing both low- and higher-frequency heating during our selected 12 h observing window.

Additionally, we note that one could, in principle, model an entire active region with only steady heating and still reproduce the distribution of observed emission measure slopes.
For example, the observed shallow slopes on periphery could be consistent with steady 1 MK, 2 MK, and 3 MK loops all emitting along the LOS.
Similarly, the steep slopes near the inner core are consistent with only steady 3 MK loops along the LOS.
This is also the case with $T_{peak}$
However, it has been exhaustively shown that truly steady heating, in which the energy deposition is constant in time, is not consistent with observed time lags or cross-correlation values \citep[e.g.][]{viall_signatures_2016}.
Thus, we do not explicitly test a steady heating model here and note that even our high-frequency heating model produces variability in the observed emission.

After the emission measure slope, $a$, the next three most important features in the classification are the maximum cross correlations for the 211-193, 193-171, and 211-171 \AA{} channel pairs.
These three channels, 211 \AA{}, 193 \AA{}, and 171 \AA{}, peak sequentially in temperature at 1.8, 1.6, and 0.8 MK, respectively (see \autoref{fig:aia-response}), suggesting that the plasma dynamics in this temperature range, which are dominated by radiative cooling and draining \citep[e.g.][]{bradshaw_cooling_2005,bradshaw_cooling_2010,bradshaw_new_2010}, are coupled to, and indicative of, the frequency at which energy is deposited in the plasma and that thermal conduction has not erased all signatures of the heating.
A strand heated by low-frequency nanoflares will be allowed to cool well below 1 MK, producing a strong cross-correlation in these channel pairs, while a strand heated by high-frequency nanoflares will rarely be allowed to cool below the equilibrium temperature such that the cross-correlation, particularly in the 171 \AA{} channel pairs, is likely to be relatively low.
This cooling behavior is illustrated for a single strand in Figure 3 of \citetalias{barnes_understanding_2019}.
The relative importance of features from these three channels is again consistent with \citet{bradshaw_patterns_2016} who found that short, but non-zero 211-193 \AA{} time lags, in combination with the prevalence of zero 171-131 \AA{} time lags, are inconsistent with low-frequency nanoflares.

Interestingly, we find that there are no channel pairs, either for the time lag or maximum cross-correlation, that include the 94 \AA{} channel in the ten most important features as shown in \autoref{tab:importance}.    
Observed time lags \citep{viall_evidence_2012,viall_survey_2017} show a transition between being dominated by the hot 94 \AA{} emission in the inner core to cool 94 \AA{}  emission in the periphery as evidenced by time lags changing from positive to negative, respectively.
Two proposed explanations for this switchover are that either impulsive heating in the cores is more energetic or it is more frequent.
The inability of the 94 \AA{} pairs to effectively discriminate between heating frequencies, as measured by the feature importance, points to the switchover being dominated by the energy rather than the frequency.
This is also confirmed by the top row of Figure 8 of \citetalias{barnes_understanding_2019} which shows the positive-negative switchover between the inner core and the periphery for all heating frequencies.
However, we note that, compared to the other \AR s studied by \citet{warren_systematic_2012} and \citetalias{viall_survey_2017}, our chosen \AR{}, NOAA 1158, is relatively cool.
In \AR s dominated by more hot 94 \AA{} emission, the 94 \AA{} channel pairs may have a higher feature importance.

While the maximum cross-correlation in the 211-193 \AA{} channel pair (see bottom row of \autoref{fig:correlations}) is very high across the whole \AR{}, the 193-171 \AA{} and 211-171 \AA{} maps (as well as the other 171 \AA{} pairs except for 171-131 \AA{}) show a comparatively low cross-correlation.
Combined with the heating frequency maps in \autoref{fig:frequency-maps} which indicate that the center of the \AR{} is consistent with high-frequency heating, this suggests that many of the loops in the core are kept from cooling much below 0.9 MK.

An important caveat to this method for systematic comparison as we have applied it here is that the random forest classifier trained on the simulated emission measure slopes, peak temperatures, time lags, and maximum cross-correlations cannot provide any assessment of the accuracy of our model as described in \citetalias{barnes_understanding_2019}.
The classifier can only say, out of the provided classes (high-, intermediate-, or low-frequency), which type of heating \textit{best} describes the data.
However, given another method for assessing the heating frequency or perhaps some alternative forward-modeling approach, a random forest classifier could be used to compare these two methods.
In this way, machine learning also provides a promising strategy for reconciling different modeling approaches.
