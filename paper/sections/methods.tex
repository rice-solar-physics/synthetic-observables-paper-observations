%%%%%%%%%%%%%%%%%%%%%%%%%%%%%%%%%%%%%%%%%%%%%%%%%%%%%%%%%%%%%%%%%%%%%%%%%%%%%%%
%                                   Data                                      %
%%%%%%%%%%%%%%%%%%%%%%%%%%%%%%%%%%%%%%%%%%%%%%%%%%%%%%%%%%%%%%%%%%%%%%%%%%%%%%%
\section{Observations and Analysis}\label{observations}

\begin{pycode}[manager_data]
manager_data = texfigure.Manager(
    pytex, './',
    python_dir='python',
    fig_dir='figures',
    data_dir='data',
    number=1,
)
\end{pycode}
\authorcomment1{Describe data collection and prep procedures; can give brief description of timelag and em procedures, mostly just pointing to Paper 1}

\begin{pycode}[manager_data]
fits_files = [
    'aia_lev1.5_20110212T153238_94_cutout.fits',
    'aia_lev1.5_20110212T153245_131_cutout.fits',
    'aia_lev1.5_20110212T153248_171_cutout.fits',
    'aia_lev1.5_20110212T153243_193_cutout.fits',
    'aia_lev1.5_20110212T153248_211_cutout.fits',
    'aia_lev1.5_20110212T153239_335_cutout.fits',
]
fig = plt.figure(figsize=texfigure.figsize(
    pytex,
    scale=1 if is_onecolumn() else 2,
    height_ratio=2/3,
    figure_width_context='columnwidth'
))
plt.subplots_adjust(hspace=0.0,wspace=0.03)
for i,f in enumerate(fits_files):
    m = Map(os.path.join(manager_data.data_dir, 'observations', f))
    m = m.submap(SkyCoord(Tx=-440*u.arcsec,Ty=-380*u.arcsec,frame=m.coordinate_frame),
                 SkyCoord(Tx=-185*u.arcsec,Ty=-125*u.arcsec,frame=m.coordinate_frame))
    ax = fig.add_subplot(2,3,i+1,projection=m)
    norm = ImageNormalize(vmin=0, vmax=m.data.max(), stretch=SqrtStretch())
    m.plot(axes=ax, title=False, annotate=False, norm=norm)
    ax.grid(alpha=0)
    lon,lat = ax.coords
    lon.set_ticks(color='k',number=4)
    lat.set_ticks(color='k',number=4)
    if i != 3:
        lon.set_ticklabel_visible(False)
        lat.set_ticklabel_visible(False)
    else:
        lat.set_ticklabel(rotation='vertical')
        lon.set_axislabel('Helioprojective Longitude')
        lat.set_axislabel('Helioprojective Latitude')
    xtext,ytext = m.world_to_pixel(SkyCoord(-420*u.arcsec, -150*u.arcsec, frame=m.coordinate_frame))
    ax.text(xtext.value, ytext.value, f'{m.meta["wavelnth"]} $\mathrm{{\AA}}$',
            color='w', fontsize=plt.rcParams['xtick.labelsize'])
fig_intensity_maps = manager_data.save_figure('intensity-maps')
fig_intensity_maps.caption = r'Maps of the observed intensity in all 6 EUV channels of AIA at a single point in time. The images of have been prepped, derotated, and cropped to \AR{} NOAA 1158.'
fig_intensity_maps.figure_env_name = 'figure*'
fig_intensity_maps.figure_width = r'\columnwidth' if is_onecolumn() else r'2\columnwidth'
fig_intensity_maps.fig_str = fig_str
\end{pycode}
\py[manager_data]|fig_intensity_maps|

%%%%%%%%%%%%%%%%%%%%%%%%%%%%%%%%%%%%%%%%%%%%%%%%%%%%%%%%%%%%%%%%%%%%%%%%%%%%%%%
%                                   EM Slopes                                 %
%%%%%%%%%%%%%%%%%%%%%%%%%%%%%%%%%%%%%%%%%%%%%%%%%%%%%%%%%%%%%%%%%%%%%%%%%%%%%%%
\subsection{Emission Measure Slopes}\label{em_slopes}

\begin{pycode}[manager_em]
manager_em = texfigure.Manager(
    pytex, './',
    python_dir='python',
    fig_dir='figures',
    data_dir='data',
    number=2,
)
from formatting import heating_palette
\end{pycode}

\authorcomment1{Describe EM results, what we did specifically to get them, show histograms of slopes as well as maps; maybe have some comparison with models here, i.e. lay histograms on top of each other}

\begin{pycode}[manager_em]
fig = plt.figure(figsize=texfigure.figsize(
    pytex,
    scale=1 if is_onecolumn() else 2,
    height_ratio=1/2,
    figure_width_context='columnwidth'
))
plt.subplots_adjust(wspace=0.31)
### Map ###
slope_map = Map(os.path.join(manager_em.data_dir, 'observations', 'em_slope.fits'))
slope_map = slope_map.submap(
    SkyCoord(Tx=-410*u.arcsec,Ty=-325*u.arcsec,frame=slope_map.coordinate_frame),
    SkyCoord(Tx=-225*u.arcsec,Ty=-150*u.arcsec,frame=slope_map.coordinate_frame))
cax = fig.add_axes([0.125, 0.82, 0.335, 0.025])
ax = fig.add_subplot(121, projection=slope_map)
im = slope_map.plot(
    axes=ax,
    cmap='viridis',
    vmin=2, vmax=5,
    title=False, annotate=False
)
ax.grid(alpha=0)
# HPC Axes
lon,lat = ax.coords
lon.set_ticks(number=4)
lat.set_ticks(number=2)
lat.set_ticklabel(rotation='vertical')
lon.set_axislabel('Helioprojective Longitude',)
lat.set_axislabel('Helioprojective Latitude',)
# Colorbar
cbar = fig.colorbar(im,cax=cax, orientation='horizontal',)
cbar.ax.xaxis.set_ticks_position('top')
cbar.set_ticks([2,3,4,5])
### Histograms ###
ax = fig.add_subplot(122,)
bins = np.arange(1,6,0.05)
colors = ['k'] + heating_palette()
# Plot Model EM Slopes
for i,h in enumerate(['observations', 'high_frequency', 'intermediate_frequency', 'low_frequency']):
    d = Map(os.path.join(manager_em.data_dir, f'{h}', 'em_slope.fits')).data.flatten()
    h,b,_ = ax.hist(
        d[~np.isnan(d)],
        bins=bins,
        histtype='step',
        density=True,
        color=colors[i],
        label=h.split('_')[0].capitalize()
    )
# Ticks and Spines
ax.set_xlim(1.5,5.5);
ax.set_xticks([2,3,4,5])
ax.spines['top'].set_visible(False)
ax.spines['right'].set_visible(False)
ax.set_yticks(ax.get_yticks()[1:-1])
ax.spines['left'].set_bounds(ax.get_yticks()[0],ax.get_yticks()[-1])
ax.spines['bottom'].set_bounds(ax.get_xticks()[0],ax.get_xticks()[-1])
# Labels and legends
ax.set_xlabel(r'$a$')
ax.set_ylabel(r'Number of Pixels (Normalized)')
ax.legend(frameon=False,loc=1)
### Save ###
fig_em_slopes = manager_em.save_figure('em-slopes')
fig_em_slopes.caption = r'Left: Map of emission measure slope as computed from the time-averaged observed intensities. Right: Histogram of observed emission measure slope with model emission measure slopes overlaid.'
fig_em_slopes.figure_env_name = 'figure*'
fig_em_slopes.figure_width = r'\columnwidth' if is_onecolumn() else r'2\columnwidth'
fig_em_slopes.fig_str = fig_str
\end{pycode}
\py[manager_em]|fig_em_slopes|

%%%%%%%%%%%%%%%%%%%%%%%%%%%%%%%%%%%%%%%%%%%%%%%%%%%%%%%%%%%%%%%%%%%%%%%%%%%%%%%
%                                   Timelags                                  %
%%%%%%%%%%%%%%%%%%%%%%%%%%%%%%%%%%%%%%%%%%%%%%%%%%%%%%%%%%%%%%%%%%%%%%%%%%%%%%%
\subsection{Timelags}\label{timelags}

\begin{pycode}[manager_timelags]
manager_timelags = texfigure.Manager(
    pytex, './',
    python_dir='python',
    fig_dir='figures',
    data_dir='data',
    number=3,
)
\end{pycode}

\begin{figure*}
    \plotone{figures/timelag_maps.pdf}
    \caption{Timelag maps as calculated from intensity data from observations of \AR{} NOAA 1158 by SDO/AIA. Timelag maps are shown for every possible channel pair as indicated in the upper left corner of each map. The colorbar range from -5000 s to +5000 s. Blues}
    \label{fig:timelag_maps}
\end{figure*}

\begin{figure*}
    \plotone{figures/correlation_maps.pdf}
    \caption{Same as \autoref{fig:timelag_maps} except here we show the maximum value of the cross-correlation as derived from the observations.}
    \label{fig:correlation_maps}
\end{figure*}

\authorcomment2{Discussion of observed timelags and cross-correlation values}

\authorcomment2{Possibly put histograms of observations versus models here too}

\authorcomment1{Emphasize the point that no single heating model is consistent with the observations. Need multiple heating frequencies; this will lead into random forest stuff}