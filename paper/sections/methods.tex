%%%%%%%%%%%%%%%%%%%%%%%%%%%%%%%%%%%%%%%%%%%%%%%%%%%%%%%%%%%%%%%%%%%%%%%%%%%%%%%
%                                   Data                                      %
%%%%%%%%%%%%%%%%%%%%%%%%%%%%%%%%%%%%%%%%%%%%%%%%%%%%%%%%%%%%%%%%%%%%%%%%%%%%%%%
\section{Observations and Analysis}\label{sec:observations}
% spell-checker: disable %
\begin{pycode}[manager_data]
manager_data = texfigure.Manager(
    pytex, './',
    python_dir='python',
    fig_dir='figures',
    data_dir='data',
    number=1,
)
\end{pycode}
% spell-checker: enable %

We analyze 12 hours of AIA observations of \AR{} NOAA 1158 in six EUV channels, 94, 131, 171, 193, 211, and 335 \AA{}, beginning at 2011 February 12 12:00:00 UTC and ending at 2011 February 13 00:00:00 UTC.
The \AR{} was chosen from the catalogue of \AR s originally compiled by \citet{warren_systematic_2012} and later studied by \citetalias{viall_survey_2017}.
The full-disk, level-1 AIA data products in FITS file format are obtained from the Joint Science Operations Center \citep[JSOC,][]{couvidat_observables_2016} at the full instrument cadence of 12 s and full spatial resolution using the drms Python client \citep{glogowski_drms_2019}.
This amounts to a total of 21597 images across all six channels and the entire 12 hour observing window.

After downloading the data, we apply the \texttt{aiaprep} method, as implemented in sunpy \citep{the_sunpy_community_sunpy_2020}, to each full-disk image in order to remove the instrument roll angle, align the center of the image with the center of the Sun, and scale each image to a common spatial resolution such that images in all channels have a spatial scale of 0.6\arcsec-per-pixel.
Additionally, we normalize each image by the exposure time such that the data have units of DN pixel$^{-1}$ s$^{-1}$.
Next, we align each image with the observation at 2011 February 12 15:33:45 UTC (the time of the original observation of NOAA 1158 by \citet{warren_systematic_2012}) by ``derotating'' each image using the Snodgrass empirical rotation rate \citep{snodgrass_magnetic_1983}.
After aligning the images in every channel to a common time, we crop each full-disk image such that the bottom left corner of the image is $(-440\arcsec,-375\arcsec)$ and the top right corner is $(-140\arcsec,-75\arcsec)$, where the two coordinates are the longitude and latitude, respectively, in the helioprojective coordinate system \citep[see][]{thompson_coordinate_2006} defined by an observer at the location of the SDO spacecraft on 2011 February 12 15:33:45.
\autoref{fig:intensity-maps} shows the level-1.5, exposure-time-normalized, derotated, and cropped AIA observations of \AR{} NOAA 1158 at 2011 February 12 15:33:45 in all six EUV channels of interest.

% spell-checker: disable %
\begin{pycode}[manager_data]
fits_files = [
    'aia_lev1.5_20110212T153238_94_cutout.fits',
    'aia_lev1.5_20110212T153245_131_cutout.fits',
    'aia_lev1.5_20110212T153248_171_cutout.fits',
    'aia_lev1.5_20110212T153243_193_cutout.fits',
    'aia_lev1.5_20110212T153248_211_cutout.fits',
    'aia_lev1.5_20110212T153239_335_cutout.fits',
]
fig = plt.figure(figsize=texfigure.figsize(
    pytex,
    scale=1 if is_onecolumn() else 2,
    height_ratio=2.25/3,
    figure_width_context='columnwidth'
))
plt.subplots_adjust(hspace=0.05,wspace=0.03)
for i,f in enumerate(fits_files):
    m = Map(os.path.join(manager_data.data_dir, 'observations', f))
    m = Map(m.data/m.exposure_time.to(u.s).value, m.meta)
    m = m.submap(SkyCoord(Tx=-440*u.arcsec,Ty=-380*u.arcsec,frame=m.coordinate_frame),
                 SkyCoord(Tx=-185*u.arcsec,Ty=-125*u.arcsec,frame=m.coordinate_frame))
    ax = fig.add_subplot(2, 3, i+1, projection=m)
    norm = ImageNormalize(vmin=0, vmax=m.data.max(), stretch=SqrtStretch())
    im = m.plot(axes=ax, title=False, annotate=False, norm=norm)
    ax.grid(alpha=0)
    lon,lat = ax.coords
    lon.set_ticks(color='k',number=4)
    lat.set_ticks(color='k',number=4)
    if i != 3:
        lon.set_ticklabel_visible(False)
        lat.set_ticklabel_visible(False)
    else:
        lat.set_ticklabel(rotation='vertical')
        lon.set_axislabel('Helioprojective Longitude')
        lat.set_axislabel('Helioprojective Latitude')
    ax.text((-420*u.arcsec).to('deg').value,
            (-150*u.arcsec).to('deg').value,
            f'{m.meta["wavelnth"]} $\mathrm{{\AA}}$',
            transform=ax.get_transform('world'),
            verticalalignment='top',
            horizontalalignment='left',
            color='w',
            fontsize=plt.rcParams['legend.fontsize'])
    pos = ax.get_position().get_points()
    cax = fig.add_axes([pos[0,0], pos[1,1]+0.0075, pos[1,0]-pos[0,0], 0.015])
    cbar = fig.colorbar(im, cax=cax, orientation='horizontal')
    cbar.locator = MaxNLocator(nbins=4, prune='lower')
    cbar.ax.tick_params(labelsize=plt.rcParams['legend.fontsize'], width=0.5)
    cbar.update_ticks()
    cbar.ax.xaxis.set_ticks_position('top')
    cbar.outline.set_linewidth(0.5)

fig_intensity_maps = manager_data.save_figure('intensity-maps')
fig_intensity_maps.caption = r'Active region NOAA 1158 as observed by AIA on 2011 February 12 15:32 UTC in the six EUV channels of interest. The data have been processed to level-1.5, aligned to the image at 2011 February 12 15:33:45 UTC, and cropped to the area surrounding NOAA 1158. The intensities are in units of DN pixel$^{-1}$ s$^{-1}$. In each image, the colorbar is on a square root scale and is normalized between zero and the maximum intensity. The color tables are the standard AIA color tables as implemented in sunpy.'
fig_intensity_maps.figure_env_name = 'figure*'
fig_intensity_maps.figure_width = r'\columnwidth' if is_onecolumn() else r'2\columnwidth'
fig_intensity_maps.fig_str = fig_str
\end{pycode}
\py[manager_data]|fig_intensity_maps|
% spell-checker: enable %

%%%%%%%%%%%%%%%%%%%%%%%%%%%%%%%%%%%%%%%%%%%%%%%%%%%%%%%%%%%%%%%%%%%%%%%%%%%%%%%
%                                   EM Slopes                                 %
%%%%%%%%%%%%%%%%%%%%%%%%%%%%%%%%%%%%%%%%%%%%%%%%%%%%%%%%%%%%%%%%%%%%%%%%%%%%%%%
\subsection{Emission Measure Slopes and Peak Temperatures}\label{sec:em_slopes}

% spell-checker: disable %
\begin{pycode}[manager_em]
manager_em = texfigure.Manager(
    pytex, './',
    python_dir='python',
    fig_dir='figures',
    data_dir='data',
    number=2,
)
import copy
from utils import heating_palette, make_slope_map
from matplotlib.lines import Line2D
\end{pycode}
% spell-checker: enable %

After prepping, aligning, and cropping all 12 hours of AIA data for all six channels, we carry out the same analysis that we applied to our predicted observations in \citetalias{barnes_understanding_2019} in order to compute the diagnostics of the heating: the emission measure slope, peak temperature, time lag, and maximum cross-correlation.
First, we compute the emission measure distribution, \dem, in each pixel of the \AR{} from the time-averaged intensities from all six channels using the regularized inversion method of \citet{hannah_differential_2012}.
As in \citetalias{barnes_understanding_2019}, we use temperature bins of width $\Delta\log T=0.1$ with the left and right edges at $10^{5.5}$ K and $10^{7.2}$ K, respectively.
The uncertainties on the intensities are estimated using the \texttt{aia\_bp\_estimate\_error.pro} procedure provided by the AIA instrument team in the SolarSoftware package \citep[SSW,][]{freeland_data_1998}.  

% spell-checker: disable %
\begin{pycode}[manager_em]
# Calculate slope maps
slope_maps = {}
for h in heating + ['observations']:
    em_threshold = 1e27 * u.cm**(-5) if h == 'observations' else 1e24 * u.cm**(-5)
    s, r2 = make_slope_map(
        EMCube.restore(os.path.join(manager_em.data_dir, h, 'em_cube.h5')),
        em_threshold=em_threshold,
        temperature_lower_bound=8e5*u.K,
    )
    mask = ~np.logical_and(r2.data >= rsquared_threshold, np.isfinite(r2.data))
    slope_maps[h] = Map(s.data, s.meta, mask=mask)

fig = plt.figure(figsize=texfigure.figsize(
    pytex,
    scale=1 if is_onecolumn() else 2,
    height_ratio=1/2,
    figure_width_context='columnwidth'
))
plt.subplots_adjust(wspace=0.31)

### Map ###
slope_map = slope_maps['observations'].submap(
    SkyCoord(Tx=-410*u.arcsec,Ty=-325*u.arcsec,frame=slope_maps['observations'].coordinate_frame),
    SkyCoord(Tx=-225*u.arcsec,Ty=-150*u.arcsec,frame=slope_maps['observations'].coordinate_frame))
ax = fig.add_subplot(121, projection=slope_map)
im = slope_map.plot(
    axes=ax,
    cmap='viridis',
    vmin=1.5,
    vmax=5.5,
    title=False,
    annotate=False
)
ax.grid(alpha=0)
# HPC Axes
lon,lat = ax.coords
lon.set_ticks(number=4)
lat.set_ticks(number=2)
lat.set_ticklabel(rotation='vertical')
lon.set_axislabel('Helioprojective Longitude',)
lat.set_axislabel('Helioprojective Latitude',)
# Mark IC and P regions
ax.text((-305*u.arcsec).to(u.deg).value,
        (-230*u.arcsec).to(u.deg).value,
        'IC',
        transform=ax.get_transform('world'),
        color='r',
        verticalalignment='center',
        horizontalalignment='center',
        fontsize=plt.rcParams['legend.fontsize'])
ax.text((-325*u.arcsec).to(u.deg).value,
        (-310*u.arcsec).to(u.deg).value,
        'P',
        transform=ax.get_transform('world'),
        color='r',
        verticalalignment='center',
        horizontalalignment='center',
        fontsize=plt.rcParams['legend.fontsize'])
# Colorbar
pos = ax.get_position().get_points()
cax = fig.add_axes([pos[0,0], pos[1,1]+0.01, pos[1,0]-pos[0,0], 0.025])
cbar = fig.colorbar(im,cax=cax, orientation='horizontal',)
cbar.ax.xaxis.set_ticks_position('top')
cbar.set_ticks([2,3,4,5])
cbar.ax.tick_params(width=0.5)
cbar.outline.set_linewidth(0.5)

### Histograms ###
ax = fig.add_subplot(122,)
bins = np.arange(0, 8, 0.05)
colors = ['k'] + heating_palette()
legend_elements = []
# Plot Model EM Slopes
for i,h in enumerate(['observations'] + heating):
    m = slope_maps[h]
    _,b,_ = ax.hist(
        m.data[~m.mask],
        bins='fd',
        histtype='step',
        density=True,
        color=colors[i],
    )
    if h == 'observations':
        a_mean = m.data[~m.mask].mean()
        a_std = m.data[~m.mask].std()
    legend_elements.append(
        Line2D([0], [0], color=colors[i], label=h.split('_')[0].capitalize())
    )
# Ticks and Spines
ax.set_xlim(1, 8);
ax.xaxis.set_major_locator(FixedLocator([2, 3, 4, 5, 6, 7]))
ax.yaxis.set_major_locator(MaxNLocator(nbins=7, prune='both'))
ax.spines['top'].set_visible(False)
ax.spines['right'].set_visible(False)
ax.spines['left'].set_bounds(ax.get_yticks()[0],ax.get_yticks()[-1])
ax.spines['bottom'].set_bounds(ax.get_xticks()[0],ax.get_xticks()[-1])
# Labels and legends
ax.set_xlabel(r'$a$')
ax.set_ylabel(r'Number of Pixels (Normalized)')
ax.legend(handles=legend_elements, frameon=False, loc=1)
### Save ###
fig_em_slopes = manager_em.save_figure('em-slopes')
fig_em_slopes.caption = r"\textit{Left:} Map of emission measure slope, $a$, in each pixel of \AR{} NOAA 1158. The \dem{} is computed from the observed AIA intensities in the six EUV channels time-averaged over the 12-hour observing window. The \dem{} in each pixel is then fit to $T^a$ over the temperature interval $8\times10^5\,\textup{K}\le T < T_{peak}$. Any pixels with $r^2<0.75$ are masked and colored white. ``IC'' and ``P'', marked in red, denote the inner core and periphery of the active region, respectively. \textit{Right:} Distribution of emission measure slopes from the left panel (black) and from \citetalias{barnes_understanding_2019} (blue, orange, green). Each histogram is normalized such that the area under the histogram is equal to 1."
fig_em_slopes.figure_env_name = 'figure*'
fig_em_slopes.figure_width = r'\columnwidth' if is_onecolumn() else r'2\columnwidth'
fig_em_slopes.fig_str = fig_str
\end{pycode}
\py[manager_em]|fig_em_slopes|
% spell-checker: enable %

The left panel of \autoref{fig:em-slopes} shows the emission measure slope, $a$, as computed from the observed emission measure distribution in each pixel of \AR{} NOAA 1158.
We calculate $a$ by fitting a first-order polynomial to the log-transformed emission measure and the temperature bin centers, $\log_{10}\textup{EM}\sim a\log_{10}T$.
As in \citetalias{barnes_understanding_2019}, the fit is only computed over the temperature range $8\times10^5\,\textup{K}\le T \le T_{peak}$, where $T_\textup{peak}=\argmax_T\,\textup{EM}(T)$ is the temperature at which the emission measure distribution peaks.
If $r^2<\py|rsquared_threshold|$ in any pixel, where $r^2$ is the correlation coefficient for the first-order polynomial fit, the pixel is masked and colored white. 

Similar to \citet{barnes_understanding_2019}, we define the \textit{inner core} as the area near the center of the \AR{} whose X-ray and EUV emission is dominated by short, closed loops.
We define the \textit{periphery} as the region farthest from the inner core containing closed loops which are still visible relative to the surrounding diffuse emission.
The inner core (``IC'') and the periphery (``P'') of the \AR{} are indicated in red in the left panel of \autoref{fig:em-slopes}.
While these definitions of different parts of the \AR{} are conceptual and qualitative rather than strictly quantitative, they will be useful in discussing our results in \autoref{sec:compare} and \autoref{sec:discussion}.

The emission measure slope tends to be more steep near the center of the \AR{} and tends to increase from $\sim2.5$ to $>5$ moving from the periphery to the inner core of the \AR{}.
This result is consistent with \citet{del_zanna_evolution_2015} who computed the emission measure slope in each pixel of \AR{} NOAA 1193 and found that $a$ was greatest near the middle of the \AR{}.
The exception to this trend is the spatially-coherent structure on the lower edge of the \AR{}, near $(-300\arcsec,-320\arcsec)$, which shows emission measure slopes $>5$.
A few regions on the top edge, near $(-350\arcsec,-140\arcsec)$, also show higher emission measure slopes.

The right panel of \autoref{fig:em-slopes} shows the distribution of emission measure slopes for every pixel in the \AR{} where $r^2\ge\py|rsquared_threshold|$.
As noted in the legend, the black histogram denotes the observed slopes while the blue, orange, and green histograms are the distributions of emission measure slopes computed from the predicted AIA intensities in \citetalias{barnes_understanding_2019} for high-, intermediate-, and low-frequency nanoflares, respectively.
The mean of the observed distribution of $a$ is \py[manager_em]|f'{a_mean:.2f}'| and the standard deviation is \py[manager_em]|f'{a_std:.2f}'|. 

We find that the observed distribution of slopes overlaps the distributions of predicted slopes for all three heating scenarios, suggesting that no single heating scenario can explain the width of the distribution and that a range of nanoflare heating frequencies is operating across the \AR.
In particular, the observed distribution of slopes overlaps quite strongly with both the intermediate- and high frequency-slopes.
Compared to the simulated distributions of $a$ for low- and intermediate-frequency heating, the observed distribution is wide with a relatively flat top between $a\approx3$ and $a\approx4$.
In contrast to all three simulated distributions, the observed slope distribution is not strongly peaked about any single value of $a$.

% spell-checker: disable %
\begin{pycode}[manager_em]
# Make peak temperature map
em_cube = EMCube.restore(os.path.join(manager_em.data_dir, 'observations', 'em_cube.h5'))
tpeak = em_cube.temperature_bin_centers[np.argmax(em_cube.as_array(), axis=2)]
mask = u.Quantity(em_cube.total_emission.data, em_cube.total_emission.meta['bunit']) < (1e27 * u.cm**(-5))
meta = copy.deepcopy(em_cube.all_meta()[0])
del meta['temp_unit']
del meta['temp_a']
del meta['temp_b']
meta['bunit'] = 'MK'
tpeak_map = Map(tpeak.to(meta['bunit']), meta, mask=mask)
tpeak_map = tpeak_map.submap(
    SkyCoord(Tx=-410*u.arcsec, Ty=-325*u.arcsec, frame=tpeak_map.coordinate_frame),
    SkyCoord(Tx=-225*u.arcsec, Ty=-150*u.arcsec, frame=tpeak_map.coordinate_frame),
)
# Setup figure
fig = plt.figure(figsize=texfigure.figsize(
    pytex,
    scale=0.5 if is_onecolumn() else 1,
    height_ratio=1,
    figure_width_context='columnwidth'
))
ax = fig.add_subplot(111, projection=tpeak_map)
im = tpeak_map.plot(
    axes=ax,
    cmap='inferno',
    vmin=1,
    vmax=4,
    title=False,
    annotate=False
)
ax.grid(alpha=0)
# HPC Axes
lon,lat = ax.coords
lon.set_ticks(number=4)
lat.set_ticks(number=2)
lat.set_ticklabel(rotation='vertical')
lon.set_axislabel('Helioprojective Longitude',)
lat.set_axislabel('Helioprojective Latitude',)
# Colorbar
pos = ax.get_position().get_points()
cax = fig.add_axes([pos[0,0], pos[1,1]+0.01, pos[1,0]-pos[0,0], 0.025])
cbar = fig.colorbar(im,cax=cax, orientation='horizontal',)
cbar.ax.xaxis.set_ticks_position('top')
cbar.ax.tick_params(width=0.5)
cbar.outline.set_linewidth(0.5)
# Save figure
tfig = manager_em.save_figure('em-tpeaks')
tfig.caption = r'Map of $T_{peak}$, the center of the temperature bin at which the emission measure distribution peaks, in MK, in each pixel of \AR{} NOAA 1158. Any pixels with total emission measure $<10^{27}$ cm$^{-5}$ are masked and colored white. The colorbar is linearly spaced between 1 and 4 MK.'
tfig.figure_width = r'0.5\columnwidth' if is_onecolumn() else r'\columnwidth'
tfig.fig_str = fig_str
\end{pycode}
\py[manager_em]|tfig|
% spell-checker: enable %

Additionally, we measure $T_{peak}$, the value at the center of the temperature bin in which the emission measure distribution is maximized, in each pixel of \AR{} NOAA 1158 from the derived emission measure distributions.
\autoref{fig:em-tpeaks} shows $T_{peak}$, in MK, in each pixel of the \AR{}.
We find that the the majority of pixels have values of $T_{peak}$ between 1.5 and 3.5 MK.
Additionally, we find that $T_{peak}$ is highest in the inner core of the \AR{}, just over 3.5 MK, and decreases to between 1.5 and 2 MK as we move outward toward the periphery.
Though not shown here, we find this same general trend in our model \AR s for all heating frequencies.
For example, Figure 5 of \citetalias{barnes_understanding_2019} shows that the \dem{} distributions near the inner core of our model \AR{} have $T_{peak}\approx3$ MK while the \dem{} closer to the periphery has $T_{peak}\approx2$ MK.

We note that there are several small regions which have $T_{peak}\ge4$ MK.
Comparing \autoref{fig:em-tpeaks} with \autoref{fig:intensity-maps}, we see that these regions correspond to ``open'' fan loops which are cooler and have lower signal-to-noise ratio, suggesting that the inverted \dem{} solutions are not reliable in these regions.
However, we already exclude the emission measure slopes in these regions due to their low correlation coefficients (see \autoref{fig:em-slopes}) and thus the unreliable inverted solutions will not affect our later predictions in \autoref{sec:compare}.

%%%%%%%%%%%%%%%%%%%%%%%%%%%%%%%%%%%%%%%%%%%%%%%%%%%%%%%%%%%%%%%%%%%%%%%%%%%%%%%
%                                   Time lags                                  %
%%%%%%%%%%%%%%%%%%%%%%%%%%%%%%%%%%%%%%%%%%%%%%%%%%%%%%%%%%%%%%%%%%%%%%%%%%%%%%%
\subsection{Time Lags}\label{sec:timelags}

% spell-checker: disable %
\begin{pycode}[manager_timelags]
manager_timelags = texfigure.Manager(
    pytex, './',
    python_dir='python',
    fig_dir='figures',
    data_dir='data',
    number=3,
)
file_format = os.path.join(manager_timelags.data_dir, 'observations', '{}_{}_{}.fits')
\end{pycode}

\begin{pycode}[manager_timelags]
aia = InstrumentSDOAIA([0,1]*u.s, None)
T = np.logspace(4,8,500) * u.K
fig = plt.figure(figsize=texfigure.figsize(
    pytex,
    scale=0.75 if is_onecolumn() else 1,
    height_ratio=2/3,
    figure_width_context='columnwidth'
))
ax = fig.gca()
pal = qualitative_palette(len(aia.channels))
for i,c in enumerate(aia.channels):
    resp = splev(T.value, c['temperature_response_spline'])
    ax.plot(T, resp/resp.max(), color=pal[i], label=r'{} $\textup{{\AA}}$'.format(c['name']))
ax.set_xlabel(r'$T\,\,[\textup{K}]$')
ax.set_ylabel(r'$K_c/\max{K_c}$')
ax.set_xscale('log')
ax.set_xlim(10**4.5,10**7.5)
ax.set_ylim(0,1.02)
ax.legend(loc='lower center', frameon=False, bbox_to_anchor=(0.5,1.02), ncol=3)
tfig = manager_timelags.save_figure('aia-response', fext='.pdf')
tfig.caption = r'Temperature response functions for all six EUV channels of AIA as a function of temperature computed by the \texttt{aia\_get\_response.pro} procedure in SSW. Each response function is normalized to its maximum.'
tfig.figure_width = r'0.75\columnwidth' if is_onecolumn() else r'\columnwidth'
tfig.fig_str = fig_str
\end{pycode}
\py[manager_timelags]|tfig|
% spell-checker: enable %

Next, we apply the time lag analysis of \citet{viall_evidence_2012} to every pixel in the \AR{} over the entire 12-hour observing window at the full temporal and spatial resolution.
As in \citetalias{barnes_understanding_2019}, we compute the cross-correlation, $\mathcal{C}_{AB}$, between all possible ``hot-cool'' pairs, $AB$, of the six EUV channels of AIA (15 in total) and find the time lag, $\tau_{AB}$, the temporal offset which maximizes the cross-correlation, in each pixel of the observed \AR{}.
We consider all possible offsets over the interval $\pm6$ hours.
Following the convention of \citet{viall_evidence_2012}, we take the order of the channels, from hottest to coolest, to be: 94, 335, 211, 193, 171, 131 \AA{}, meaning that \textit{a positive time lag indicates cooling plasma}.
Observationally, this is often a good representation of how the plasma in a quiescent \AR{} evolves through the AIA channels.
The response curves as a function of temperature for these six EUV channels of AIA are shown in \autoref{fig:aia-response}.
The details of the cross-correlation and time lag calculations can be found in the appendix of \citetalias{barnes_understanding_2019}.

% spell-checker: disable %
\begin{pycode}[manager_timelags]
fig = plt.figure(figsize=texfigure.figsize(
    pytex,
    scale=1 if is_onecolumn() else 2,
    height_ratio=5/3,
    figure_width_context='columnwidth'
))
plot_params = {
    'title': False, 
    'annotate': False,
    'vmin': -(6e3*u.s).to(u.s).value,
    'vmax': (6e3*u.s).to(u.s).value,
    'cmap': 'idl_bgry_004',
}
axes = []
for i,cp in enumerate(channel_pairs):
    m = Map(file_format.format('timelag', *cp))
    mc = Map(file_format.format('correlation', *cp))
    m = Map(m.data, m.meta, mask=mc.data<=correlation_threshold)
    m = m.submap(SkyCoord(Tx=-440*u.arcsec,Ty=-380*u.arcsec,frame=m.coordinate_frame),
                 SkyCoord(Tx=-185*u.arcsec,Ty=-125*u.arcsec,frame=m.coordinate_frame))
    ax = fig.add_subplot(5, 3, i+1, projection=m)
    axes.append(ax)
    im = m.plot(axes=ax, **plot_params)
    ax.grid(alpha=0)
    lon = ax.coords[0]
    lat = ax.coords[1]
    lon.set_ticks(number=4)
    lat.set_ticks(number=4) 
    if i == 12:
        lat.set_ticklabel(rotation='vertical',)
        lat.set_axislabel(r'Helioprojective Latitude',)
        lon.set_axislabel(r'Helioprojective Longitude',)
    else:
        lon.set_ticklabel_visible(False)
        lat.set_ticklabel_visible(False)
    ax.text(
        (-190*u.arcsec).to('deg').value,
        (-360*u.arcsec).to('deg').value,
        #(-430*u.arcsec).to('deg').value,
        #(-135*u.arcsec).to('deg').value,
        '{}-{}'.format(*cp),
        color='tab:pink' if i in [7,8,10,11,12,13] else 'k',#'tab:pink',
        transform=ax.get_transform('world'),
        fontsize=plt.rcParams['axes.labelsize'],
        horizontalalignment='right',
        verticalalignment='bottom',
    )
plt.subplots_adjust(wspace=0.03, hspace=0.03)
cax = fig.add_axes([
    axes[2].get_position().get_points()[1,0] + 0.01,
    axes[-1].get_position().get_points()[0,1] ,
    0.02,
    axes[2].get_position().get_points()[1,1] - axes[-1].get_position().get_points()[0,1], 
])
cbar = fig.colorbar(im, cax=cax, orientation='vertical')
cbar.ax.yaxis.set_ticks_position('right')
cbar.ax.tick_params(width=0.5)
cbar.outline.set_linewidth(0.5)
fig_timelags = manager_timelags.save_figure('timelags')
fig_timelags.caption = r'Time lag maps of \AR{} NOAA 1158 for all 15 channel pairs. The value of each pixel indicates the temporal offset, in seconds, which maximizes the cross-correlation \citepalias[see appendix of][]{barnes_understanding_2019}. The range of the colorbar is $\pm6000$ s. If $\max\mathcal{C}_{AB}<0.1$, the pixel is masked and colored white. Each map has been cropped to emphasize the core of the \AR{} such that the bottom left corner and top right corner of each image correspond to $(-440\arcsec,-380\arcsec)$ and $(-185\arcsec,-125\arcsec)$, respectively.'
fig_timelags.figure_env_name = 'figure*'
fig_timelags.figure_width = r'\columnwidth' if is_onecolumn() else r'2\columnwidth'
fig_timelags.fig_str = fig_str
\end{pycode}
\py[manager_timelags]|fig_timelags|
% spell-checker: enable %

\autoref{fig:timelags} shows the time-lag maps of \AR{} NOAA 1158 for all 15 channel pairs.
Blacks, blues, and greens indicate negative time lags while reds, oranges, and yellows correspond to positive time lags.
Olive green denotes near zero time lag.
The range of the colorbar is $\pm6000$ s.
If the maximum cross-correlation in a given pixel is too small, $\max\mathcal{C}_{AB}<\py|correlation_threshold|$, the pixel is masked and colored white.

Note that \citetalias{viall_survey_2017} carried out the time lag analysis on this same \AR{}, NOAA 1158 (their region 2), as part of a survey of the catalogue of \AR{}s compiled by \citet{warren_systematic_2012}.
We repeat this analysis here to ensure that we are treating the observed intensities in the exact same manner as the predicted intensities from \citetalias{barnes_understanding_2019}.
The method employed here for calculating the time lag \citepalias[see Appendix C of][]{barnes_understanding_2019} yields quantitatively identical results to that used by \citetalias{viall_survey_2017} \citep[see Section 2 of][]{viall_evidence_2012}.
Comparing \autoref{fig:timelags} to Figure 2 and Figure 4 of \citetalias{viall_survey_2017}, we find all of the same qualitative features in the time lag maps for each channel pair.
We note that, while there are differences between the two sets of time lag maps, these can be attributed to differences in the field of view of the cutouts and ``derotation'' reference date.
Additionally, \citetalias{viall_survey_2017} did not apply any masking based on the maximum cross-correlation value.

For the majority of the channel pairs, we find persistent positive time lags across most of the \AR{}, indicative of plasma cooling through the AIA passbands.
The 94-131, 211-131, 193-171, and 193-131 \AA{} pairs show coherent positive time lags on the periphery of the \AR{}, but zero time lag in the center of the \AR{}.
On the other hand, 171-131 \AA{} channel pair map shows zero time lags in nearly every pixel of the \AR{}.
While the 211-193 \AA{} channel pair map also appears to show mostly zero time lags, there are a significant number of positive time lags compared to the 171-131 \AA{} channel pair, consistent with \citet{viall_evidence_2012,viall_survey_2017}.
From \autoref{fig:aia-response}, we see that both of these channel pairs are strongly overlapping in temperature space such that their respective peaks in intensity are likely to be close to coincident in time as the plasma cools.
However, the presence of the positive, though small 211-193 \AA{} time lags compared to the zero 171-131 \AA{} time lags is indicative of plasma cooling into, but not through the 131 \AA{} channel \citep{bradshaw_patterns_2016}.

Additionally, the 94-335, 94-193, and 94-211 \AA{} pairs all show significant coherent negative time lags.
Because the 94 \AA{} channel is bimodal in temperature (see \autoref{fig:aia-response}), a negative time lag is indicative of the plasma cooling first through the ``cool'' channel and then through the cooler component of the 94 \AA{} bandpass.
In this case, the cooler, 1 MK component of the 94 \AA{} dominates the emission.
The 94-171 and 94-131 pairs show only positive time lags because the 171 \AA{} and 131 \AA{} channels peak at cooler temperatures than the cool component of the 94 \AA{} channel.
See \citetalias{viall_survey_2017} for a more detailed discussion of the time lag results from NOAA 1158.
Note that unlike the predicted time lags in \citetalias{barnes_understanding_2019}, none of the pairs involving the 131 \AA{} channel, which is also bimodal in temperature, show any coherent negative time lags and, in particular, the inner cores of each 131 \AA{} pair show zero time lag.
This is indicative of an excess of hot plasma in our model \AR{} relative to the observations.

% spell-checker: disable %
\begin{pycode}[manager_timelags]
fig = plt.figure(figsize=texfigure.figsize(
    pytex,
    scale=1 if is_onecolumn() else 2,
    height_ratio=5/3,
    figure_width_context='columnwidth'
))
plot_params = {
    'title': False, 
    'annotate': False,
    'vmin': 0,
    'vmax': 1,
    'cmap': 'magma',
}
axes = []
for i,cp in enumerate(channel_pairs):
    m = Map(file_format.format('correlation', *cp))
    m = m.submap(SkyCoord(Tx=-440*u.arcsec,Ty=-380*u.arcsec,frame=m.coordinate_frame),
                 SkyCoord(Tx=-185*u.arcsec,Ty=-125*u.arcsec,frame=m.coordinate_frame))
    ax = fig.add_subplot(5, 3, i+1, projection=m)
    axes.append(ax)
    im = m.plot(axes=ax, **plot_params)
    ax.grid(alpha=0)
    lon = ax.coords[0]
    lat = ax.coords[1]
    lon.set_ticks(number=4)
    lat.set_ticks(number=4) 
    if i == 12:
        lat.set_ticklabel(rotation='vertical',)
        lat.set_axislabel(r'Helioprojective Latitude',)
        lon.set_axislabel(r'Helioprojective Longitude',)
    else:
        lon.set_ticklabel_visible(False)
        lat.set_ticklabel_visible(False)
    ax.text(
        (-430*u.arcsec).to('deg').value,
        (-135*u.arcsec).to('deg').value,
        '{}-{}'.format(*cp),
        transform=ax.get_transform('world'),
        color='w',
        fontsize=plt.rcParams['axes.labelsize'],
        horizontalalignment='left',
        verticalalignment='top',
    )
plt.subplots_adjust(wspace=0.03, hspace=0.03)
cax = fig.add_axes([
    axes[2].get_position().get_points()[1,0] + 0.01,
    axes[-1].get_position().get_points()[0,1] ,
    0.02,
    axes[2].get_position().get_points()[1,1] - axes[-1].get_position().get_points()[0,1], 
])
cbar = fig.colorbar(im, cax=cax, orientation='vertical')
cbar.ax.yaxis.set_ticks_position('right')
cbar.ax.tick_params(width=0.5)
cbar.outline.set_linewidth(0.5)
fig_correlations = manager_timelags.save_figure('correlations')
fig_correlations.caption = r'Same as \autoref{fig:timelags} except here we show the maximum value of the cross-correlation, $\max\mathcal{C}_{AB}$, for each channel pair.'
fig_correlations.figure_env_name = 'figure*'
fig_correlations.figure_width = r'\columnwidth' if is_onecolumn() else r'2\columnwidth'
fig_correlations.fig_str = fig_str
\end{pycode}
\py[manager_timelags]|fig_correlations|
% spell-checker: enable %

\autoref{fig:correlations} shows the maximum cross-correlation, $\max\mathcal{C}_{AB}$, in each pixel of the \AR{}.
In this figure, we do not mask any of the pixels. Though the value of the cross-correlation can range from $-1$ (perfectly anti-correlated) to $+1$ (perfectly correlated), the colorbar only ranges from 0 to 1 as we are only interested in whether the light curves in each channel pair are in phase.
I went back and looked at the distributions of max cross-correlation values for each channel pair an compared them to those shown in \citet{viall_evidence_2012} and they are all qualitatively similar.
In practice, these values are rarely less than zero as the time lag method finds the maximum cross-correlation over all possible time lags.

In every channel pair, we find that the maximum cross-correlation maps reveal coherent loop-like structures similar to those seen in the observed intensities shown in \autoref{fig:intensity-maps}, indicating that these loops and the surrounding diffuse emission are evolving coherently through the AIA passbands.
In most channel pairs, the inner core tends to have the highest cross-correlation while areas near the corners of the images have low cross-correlation.
In general, this is expected as the periphery has less emission than the core in all channels and thus lower count rates.
This lower signal-to-noise ratio means that any physical variations are more likely to be lost in the noise.
Note that the channel pairs which had the most zero time lags in \autoref{fig:timelags}, 211-193 and 171-131, show high cross-correlations across the entire \AR{}.
This again emphasizes the point that zero time lags do not correspond to steady heating.
If the whole \AR{} was heated producing steady emission, we would expect low cross-correlation and no preferred time lag due to the photon noise dominating the variability in each channel \citep{viall_signatures_2016}.
