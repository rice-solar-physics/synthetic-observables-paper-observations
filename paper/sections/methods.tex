%%%%%%%%%%%%%%%%%%%%%%%%%%%%%%%%%%%%%%%%%%%%%%%%%%%%%%%%%%%%%%%%%%%%%%%%%%%%%%%
%                                   Data                                      %
%%%%%%%%%%%%%%%%%%%%%%%%%%%%%%%%%%%%%%%%%%%%%%%%%%%%%%%%%%%%%%%%%%%%%%%%%%%%%%%
\section{Observations and Analysis}\label{observations}

\begin{pycode}[manager_data]
manager_data = texfigure.Manager(
    pytex, './',
    python_dir='python',
    fig_dir='figures',
    data_dir='data',
    number=1,
)
\end{pycode}

\authorcomment1{Describe data collection and prep procedures; can give brief description of timelag and em procedures, mostly just pointing to Paper 1}

\begin{figure*}
    \plotone{figures/intensity_maps.pdf}
    \caption{Maps of the observed intensity in all 6 EUV channels of AIA at a single point in time. The images of have been prepped, derotated, and cropped to \AR{} NOAA 1158.}
    \label{fig:intensity_maps}
\end{figure*}


%%%%%%%%%%%%%%%%%%%%%%%%%%%%%%%%%%%%%%%%%%%%%%%%%%%%%%%%%%%%%%%%%%%%%%%%%%%%%%%
%                                   EM Slopes                                 %
%%%%%%%%%%%%%%%%%%%%%%%%%%%%%%%%%%%%%%%%%%%%%%%%%%%%%%%%%%%%%%%%%%%%%%%%%%%%%%%
\subsection{Emission Measure Slopes}\label{em_slopes}

\begin{pycode}[manager_em]
manager_em = texfigure.Manager(
    pytex, './',
    python_dir='python',
    fig_dir='figures',
    data_dir='data',
    number=2,
)
from formatting import heating_palette
\end{pycode}

\authorcomment1{Describe EM results, what we did specifically to get them, show histograms of slopes as well as maps; maybe have some comparison with models here, i.e. lay histograms on top of each other}

\begin{figure}
    \plotone{figures/em_slope_map.pdf}
    \caption{Map of emission measure slope as computed from the time-averaged observed intensities.}
    \label{fig:em_slope_map}
\end{figure}

\begin{figure}
    \plotone{figures/em_slope_histograms.pdf}
    \caption{Histogram of observed emission measure slope with model emission measure slopes overlaid.}
    \label{fig:em_slope_histogram}
\end{figure}

%%%%%%%%%%%%%%%%%%%%%%%%%%%%%%%%%%%%%%%%%%%%%%%%%%%%%%%%%%%%%%%%%%%%%%%%%%%%%%%
%                                   Timelags                                  %
%%%%%%%%%%%%%%%%%%%%%%%%%%%%%%%%%%%%%%%%%%%%%%%%%%%%%%%%%%%%%%%%%%%%%%%%%%%%%%%
\subsection{Timelags}\label{timelags}

\begin{pycode}[manager_timelags]
manager_timelags = texfigure.Manager(
    pytex, './',
    python_dir='python',
    fig_dir='figures',
    data_dir='data',
    number=3,
)
\end{pycode}

\begin{figure*}
    \plotone{figures/timelag_maps.pdf}
    \caption{Timelag maps as calculated from intensity data from observations of \AR{} NOAA 1158 by SDO/AIA. Timelag maps are shown for every possible channel pair as indicated in the upper left corner of each map. The colorbar range from -5000 s to +5000 s. Blues}
    \label{fig:timelag_maps}
\end{figure*}

\begin{figure*}
    \plotone{figures/correlation_maps.pdf}
    \caption{Same as \autoref{fig:timelag_maps} except here we show the maximum value of the cross-correlation as derived from the observations.}
    \label{fig:correlation_maps}
\end{figure*}

\authorcomment2{Discussion of observed timelags and cross-correlation values}

\authorcomment2{Possibly put histograms of observations versus models here too}

\authorcomment1{Emphasize the point that no single heating model is consistent with the observations. Need multiple heating frequencies; this will lead into random forest stuff}