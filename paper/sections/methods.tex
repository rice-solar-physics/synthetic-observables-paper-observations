%%%%%%%%%%%%%%%%%%%%%%%%%%%%%%%%%%%%%%%%%%%%%%%%%%%%%%%%%%%%%%%%%%%%%%%%%%%%%%%
%                                   Data                                      %
%%%%%%%%%%%%%%%%%%%%%%%%%%%%%%%%%%%%%%%%%%%%%%%%%%%%%%%%%%%%%%%%%%%%%%%%%%%%%%%
\section{Observations and Analysis}\label{observations}

\begin{pycode}[manager_data]
manager_data = texfigure.Manager(
    pytex, './',
    python_dir='python',
    fig_dir='figures',
    data_dir='data',
    number=1,
)
\end{pycode}
\authorcomment1{Describe data collection and prep procedures; can give brief description of timelag and em procedures, mostly just pointing to Paper 1}

\begin{pycode}[manager_data]
fits_files = [
    'aia_lev1.5_20110212T153238_94_cutout.fits',
    'aia_lev1.5_20110212T153245_131_cutout.fits',
    'aia_lev1.5_20110212T153248_171_cutout.fits',
    'aia_lev1.5_20110212T153243_193_cutout.fits',
    'aia_lev1.5_20110212T153248_211_cutout.fits',
    'aia_lev1.5_20110212T153239_335_cutout.fits',
]
fig = plt.figure(figsize=texfigure.figsize(
    pytex,
    scale=1 if is_onecolumn() else 2,
    height_ratio=2/3,
    figure_width_context='columnwidth'
))
plt.subplots_adjust(hspace=0.03,wspace=0.03)
for i,f in enumerate(fits_files):
    m = Map(os.path.join(manager_data.data_dir, 'observations', f))
    m = m.submap(SkyCoord(Tx=-440*u.arcsec,Ty=-380*u.arcsec,frame=m.coordinate_frame),
                 SkyCoord(Tx=-185*u.arcsec,Ty=-125*u.arcsec,frame=m.coordinate_frame))
    ax = fig.add_subplot(2,3,i+1,projection=m)
    norm = ImageNormalize(vmin=0, vmax=m.data.max(), stretch=SqrtStretch())
    m.plot(axes=ax, title=False, annotate=False, norm=norm)
    ax.grid(alpha=0)
    lon,lat = ax.coords
    lon.set_ticks(color='k',number=3)
    lat.set_ticks(color='k',number=3)
    if i < 3:
        lon.set_ticklabel_visible(False)
    if i%3 != 0:
        lat.set_ticklabel_visible(False)
    else:
        lat.set_ticklabel(rotation='vertical')
    xtext,ytext = m.world_to_pixel(SkyCoord(-420*u.arcsec, -150*u.arcsec, frame=m.coordinate_frame))
    ax.text(xtext.value, ytext.value, f'{m.meta["wavelnth"]} $\mathrm{{\AA}}$',
            color='w', fontsize=plt.rcParams['xtick.labelsize'])
    if i == 3:
        lon.set_axislabel('Helioprojective Longitude [arcsec]')
        lat.set_axislabel('Helioprojective Latitude [arcsec]')
fig_intensity_maps = manager_data.save_figure('intensity-maps')
fig_intensity_maps.caption = r'Maps of the observed intensity in all 6 EUV channels of AIA at a single point in time. The images of have been prepped, derotated, and cropped to \AR{} NOAA 1158.'
fig_intensity_maps.figure_env_name = 'figure*'
fig_intensity_maps.figure_width = r'\columnwidth' if is_onecolumn() else r'2\columnwidth'
fig_intensity_maps.fig_str = fig_str
\end{pycode}
\py[manager_data]|fig_intensity_maps|

%%%%%%%%%%%%%%%%%%%%%%%%%%%%%%%%%%%%%%%%%%%%%%%%%%%%%%%%%%%%%%%%%%%%%%%%%%%%%%%
%                                   EM Slopes                                 %
%%%%%%%%%%%%%%%%%%%%%%%%%%%%%%%%%%%%%%%%%%%%%%%%%%%%%%%%%%%%%%%%%%%%%%%%%%%%%%%
\subsection{Emission Measure Slopes}\label{em_slopes}

\begin{pycode}[manager_em]
manager_em = texfigure.Manager(
    pytex, './',
    python_dir='python',
    fig_dir='figures',
    data_dir='data',
    number=2,
)
from formatting import heating_palette
\end{pycode}

\authorcomment1{Describe EM results, what we did specifically to get them, show histograms of slopes as well as maps; maybe have some comparison with models here, i.e. lay histograms on top of each other}

\begin{figure}
    \plotone{figures/em_slope_map.pdf}
    \caption{Map of emission measure slope as computed from the time-averaged observed intensities.}
    \label{fig:em_slope_map}
\end{figure}

\begin{figure}
    \plotone{figures/em_slope_histograms.pdf}
    \caption{Histogram of observed emission measure slope with model emission measure slopes overlaid.}
    \label{fig:em_slope_histogram}
\end{figure}

%%%%%%%%%%%%%%%%%%%%%%%%%%%%%%%%%%%%%%%%%%%%%%%%%%%%%%%%%%%%%%%%%%%%%%%%%%%%%%%
%                                   Timelags                                  %
%%%%%%%%%%%%%%%%%%%%%%%%%%%%%%%%%%%%%%%%%%%%%%%%%%%%%%%%%%%%%%%%%%%%%%%%%%%%%%%
\subsection{Timelags}\label{timelags}

\begin{pycode}[manager_timelags]
manager_timelags = texfigure.Manager(
    pytex, './',
    python_dir='python',
    fig_dir='figures',
    data_dir='data',
    number=3,
)
\end{pycode}

\begin{figure*}
    \plotone{figures/timelag_maps.pdf}
    \caption{Timelag maps as calculated from intensity data from observations of \AR{} NOAA 1158 by SDO/AIA. Timelag maps are shown for every possible channel pair as indicated in the upper left corner of each map. The colorbar range from -5000 s to +5000 s. Blues}
    \label{fig:timelag_maps}
\end{figure*}

\begin{figure*}
    \plotone{figures/correlation_maps.pdf}
    \caption{Same as \autoref{fig:timelag_maps} except here we show the maximum value of the cross-correlation as derived from the observations.}
    \label{fig:correlation_maps}
\end{figure*}

\authorcomment2{Discussion of observed timelags and cross-correlation values}

\authorcomment2{Possibly put histograms of observations versus models here too}

\authorcomment1{Emphasize the point that no single heating model is consistent with the observations. Need multiple heating frequencies; this will lead into random forest stuff}