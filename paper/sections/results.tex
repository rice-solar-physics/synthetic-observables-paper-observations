%%%%%%%%%%%%%%%%%%%%%%%%%%%%%%%%%%%%%%%%%%%%%%%%%%%%%%%%%%%%%%%%%%%%%%%%%%%%%%%
%                                   Comparisons                               %
%%%%%%%%%%%%%%%%%%%%%%%%%%%%%%%%%%%%%%%%%%%%%%%%%%%%%%%%%%%%%%%%%%%%%%%%%%%%%%%
\section{Classification Model}\label{compare}

\begin{pycode}[manager_ml]
manager_ml = texfigure.Manager(
    pytex, './',
    python_dir='python',
    fig_dir='figures',
    data_dir='data',
    number=4,
)
from formatting import heating_palette, heating_cmap
\end{pycode}

\authorcomment1{Describe random forest technique; how is data prepared; what is the RF actually doing; Description should be very detailed}

\authorcomment1{Run classifier for EM, timelag+correlation, EM+timelag+correlation; compare results; show heating frequency maps, probability maps, and stack plots for all three cases}

\begin{figure*}
    \plotone{figures/probability_maps.pdf}
    \caption{Foo bar}
    \label{fig:probability_maps}
\end{figure*}

\authorcomment1{Describe and interpret the results of the classification}

\begin{figure}
    \plotone{figures/frequency_map.pdf}
    \caption{foo bar}
    \label{fig:frequency_map}
\end{figure}
